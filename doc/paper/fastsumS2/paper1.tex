%=============================================================================
%\documentclass[11pt,twoside]{article}
\documentclass[11pt,a4paper,twoside,bibtotoc]{scrartcl}
%\documentclass[twoside,11pt]{article}
%\usepackage{jnm}
%=============================================================================
\usepackage{a4wide}
\usepackage{amsfonts}
\usepackage{amsmath}
\usepackage{theorem}
%\usepackage{jkmath}
\usepackage{subfigure}
\usepackage{natbib}
\usepackage{algorithm}
\usepackage{algorithmic}
\usepackage{graphicx}

\theoremstyle{plain}
\newtheorem{theorem}{Theorem}[section]
\newtheorem{corollary}[theorem]{Corollary}
\newtheorem{lemma}[theorem]{Lemma}
\newtheorem{proposition}[theorem]{Proposition}
\theoremstyle{definition}
\newtheorem{definition}[theorem]{Definition}
\newtheorem{example}[theorem]{Example}
\theoremstyle{remark}
\newtheorem{remark}[theorem]{Remark}
\newenvironment{proof}{{\bf Proof.}}{$\Box$}

\newcommand{\adj}{{\vdash \hspace*{-1.72mm} \dashv}}
\newcommand{\supp}{\:{\rm supp}}

\renewcommand{\topfraction}{1}
\renewcommand{\textfraction}{0}
\setcounter{totalnumber}{4}

%============================================================================

\newcommand{\N}{\ensuremath{\mathbb{N}}}
\newcommand{\NZ}{\ensuremath{\mathbb{N}_{0}}}
\newcommand{\Z}{\ensuremath{\mathbb{Z}}}
\newcommand{\Q}{\ensuremath{\mathbb{Q}}}
\newcommand{\R}{\ensuremath{\mathbb{R}}}
\newcommand{\Rp}{\ensuremath{\mathbb{R}^{+}}}
\newcommand{\Rn}{\ensuremath{\mathbb{R}^n}}
\newcommand{\Rnn}{\ensuremath{\mathbb{R}^{n \times n}}}
\newcommand{\C}{\ensuremath{\mathbb{C}}}
\newcommand{\Pol}{\ensuremath{\Pi}}

\newcommand{\norm}[1]{\ensuremath{\left\|#1\right\|}}
\newcommand{\abs}[1]{\ensuremath{\left\vert#1\right\vert}}
\newcommand{\set}[1]{\ensuremath{\left\{#1\right\rbrace}}
\newcommand{\pset}[3]{\ensuremath{\left\{#1\ \left#2\ #3\right.\right\rbrace}}
\newcommand{\veps}{\ensuremath{\varepsilon}}
\newcommand{\vtheta}{\ensuremath{\vartheta}}
\newcommand{\vphi}{\ensuremath{\varphi}}
\newcommand{\towards}{\ensuremath{\longrightarrow}}

\newcommand{\twosphere}{\ensuremath{\mathbb{S}^2}}
\newcommand{\Ln}[2]{\ensuremath{\text{\rm{L}}^{#1}\left(#2\right)}}
\newcommand{\interv}[4]{\ensuremath{\left#1\left.#2,#3\right#4\right.}}
\newcommand{\fun}[2]{\ensuremath{#1{\hspace{-0.4ex}}\left(#2\right)}}
\newcommand{\paren}[1]{\ensuremath{\left(#1\right)}}
\newcommand{\encl}[3]{\ensuremath{\left#1#2\right#3}}
\newcommand{\bigo}[1]{\ensuremath{\mathcal{O}\paren{#1}}}
\newcommand{\smallo}[1]{\ensuremath{\mathcal{o}\paren{#1}}}
\newcommand{\jkSpacer}{\ensuremath{{}^{}}}
\newcommand{\scalarproduct}[2]{\ensuremath{\left<#1,#2\right>}}
\newcommand{\mb}[1]{\mathbf{#1}}
\newcommand{\V}[1]{\mb{#1}}
\newcommand{\transp}{\text{\rm{T}}}
\newcommand{\h}{\text{\rm{H}}}
\newcommand{\dx}{\text{\rm{d}}}
\renewcommand{\Re}{\text{\rm{Re}}}
\renewcommand{\Im}{\text{\rm{Im}}}
\newcommand{\e}{\mbox{\rm{e}}}
\newcommand{\im}{\mbox{\scriptsize\rm{i}}}
\newcommand{\diag}{\text{\rm{diag}}}
\def\invisible#1{\textcolor{white}{#1}}
\newcommand{\ceil}[1]{\encl{\lceil}{#1}{\rceil}}
\newcommand{\floor}[1]{\encl{\lfloor}{#1}{\rfloor}}

%============================================================================

\numberwithin{equation}{section}
\numberwithin{table}{section}
\numberwithin{figure}{section}

%\newlength{\temp}
%\setcounter{totalnumber}{10}
%\setcounter{topnumber}{10}

%============================================================================

\title{
%{\rm\normalsize Short Note}\\
Fast Summation on the sphere}

\date{\today}

\author{
Jens Keiner\thanks{keiner@math.uni-luebeck.de, University of
  L\"ubeck, Institute of Mathematics, D--23560 L\"ubeck} \and
Stefan Kunis\thanks{kunis@math.uni-luebeck.de, University of
  L\"ubeck, Institute of Mathematics, D--23560 L\"ubeck} \and
  Daniel Potts\thanks{potts@math.uni-luebeck.de, University of
  L\"ubeck, Institute of Mathematics, D--23560 L\"ubeck} 
}

%=============================================================================
\begin{document}
\maketitle

\begin{abstract}
\medskip

%\noindent
%2000 {\it Mathematics Subject Classification}. 65F10, 65F15, 65T40.

\noindent
{\it Key words and phrases}.  
\end{abstract}

%-----------------------------------------------------------------------------
\section{Introduction}
\label{sect:1}
Radial basis functions are a powerful tool in many areas of multi-dimensional 
approximation and interpolation.
In radial basis function methods one approximates functions from $\R^3
\rightarrow \R$ by linear combinations of translates of a single radial 
symmetric function $\phi:\R^3 \rightarrow \R, \, \phi(x)=\phi(\|x\|_2)$.
The spherical counterpart are the \emph{zonal} functions which depent solely
on the geodesic distance of two points on the sphere $\twosphere:=\{
\V{\xi}: \|\V{\xi}\|_2=1\} \subset \R^3$ and the notion of the former
translation is replaced by the usual inner product.
More formally, let $K \in \Ln{2}{\interv{[}{-1}{1}{]}}$ and define for fixed
$\V{\eta} \in \twosphere$ the $\V{\eta}$-zonal function 
\[
  \fun{K}{\V{\eta} \: \cdot}: \twosphere \rightarrow \R,\ \V{\xi} \mapsto
  \fun{K}{\V{\eta} \cdot \V{\xi}}\,.% \qquad \V{\xi} \in \twosphere.
\]
Of course, every radial function corresponds to a zonal function by means of
\[
  \fun{K}{\V{\eta} \cdot \V{\xi}} = \phi\left(\sqrt{2-2\V{\eta} \cdot
  \V{\xi}}\right)\,.
\]

\section{Prerequisites}
\label{sect:2}
The Legendre polynomials $P_k : \interv{[}{-1}{1}{]} \rightarrow \R$, $k \in
\N_{0}$ as classical orthogonal polynomials are given by their corresponding
\emph{Rodrigues formula}
\[
\fun{P_k}{x} := \frac{1}{2^k k!} \frac{\dx^k}{\dx x^k} \paren{x^2-1}^k.
\]
One verifies $\fun{P_{k}}{\pm1} = \paren{\pm1}^{k}$, $\fun{P_{k}}{\cos\theta}
\le \sqrt{\frac{2}{\pi k \sin\theta}}$ for $\theta \in (0,\pi)$, $k \ge 1$, 
and $\max_{x \in \interv{[}{-1}{1}{]}} \abs{\fun{P_{k}}{x}} = 1$, 
\cite[pp. 47]{niuv}.
Furthermore, two recurrence relations are given by
\begin{equation}\label{three1}
\paren{k+1}\fun{P_{k+1}}{x} = \paren{2k+1}x\fun{P_{k}}{x} - k\fun{P_{k-1}}{x}
\end{equation}
and
\begin{equation}\label{three2}
\paren{2k+1} \fun{P_{k}}{x} = \fun{P_{k+1}'}{x} - \fun{P_{k-1}'}{x}.
\end{equation}

Let the space of real-valued continuous functions on the sphere be decomposed
into the direct sum of spaces of spherical harmonics, i.e.,
$C(\twosphere)=\bigoplus_{k=0}^{\infty} \mathcal{H}_k$ \textbf{Warum $C$ und 
nicht $L^2$?}, and let $\set{Y_{k}^n}_{k \in \NZ; n=-k,\ldots,k}$ denote the 
standard orthonormal \textbf{ja, bez\"uglich $L^2$-Skalarprodukt} basis of 
spherical harmonics given by
\[
  \fun{Y_{k}^n}{\V{\xi}} = \fun{Y_{k}^n}{\theta,\varphi} = 
  \sqrt{\frac{2k+1}{4\pi}} 
  \fun{P_{k}^{|n|}}{\cos\theta} \e^{\im n \varphi}.
\]
The orthogonal expansion of the zonal function $\fun{K}{\V{\eta} \: \cdot}$
in terms of Legendre-polynomials is
\begin{equation}
  \label{equation:kernelExpansion}
  \fun{K}{\V{\eta} \cdot \V{\xi}} = \sum_{k = 0}^{\infty} \frac{2k+1}{4\pi} 
  \fun{P_{k}}{\V{\eta} \cdot \V{\xi}},
\end{equation}
where the \emph{Legendre transform}, i.e. the \emph{symbol} of $K$, is given
for $k \in \NZ$ by
\[
  \fun{K^{\wedge}}{k} := 2 \pi \int_{-1}^{1} \fun{K}{x} \fun{P_{k}}{x} \dx 
  x\,.
\]
Using the addition theorem
\[
\sum_{n=-k}^{k} \fun{Y_{k}^n}{\V{\xi}} \overline{\fun{Y_{k}^n}{\V{\eta}}} =
    \frac{2k+1}{4\pi}\fun{P_k}{\V{\eta} \cdot \V{\xi}}
\]
the expansion \eqref{equation:kernelExpansion} of 
$\fun{K}{\V{\eta} \: \cdot}$ transforms to
\begin{equation}
  \label{Basics:Kernel}
  \fun{K}{\V{\eta} \cdot \V{\xi}} = \sum_{k = 0}^{\infty} \sum_{n=-k}^k
  \fun{K^{\wedge}}{k}\overline{\fun{Y_{k}^n}{\V{\eta}}} \fun{Y_{k}^n}{\V{\xi}}.
\end{equation}

The convolution of two zonal functions is considered in the following lemma.
\begin{lemma} {\bf Normalisierung!!}
  Let $Q,P\in \Ln{2}{\interv{[}{-1}{1}{]}}$ and
  \[
  \fun{K}{\V{\eta} \cdot \V{\xi}} := \fun{\left(Q * P\right)}{\V{\eta} \cdot
    \V{\xi}},
  \]
  where
  \[
    \fun{\left(Q * P\right)}{\V{\eta} \cdot \V{\xi}}:= 
    \int_{\twosphere} \fun{Q}{\V{\eta} \cdot \V{\nu}}
    \fun{P}{\V{\nu} \cdot \V{\xi}} \dx \omega\left(\nu\right)
  \]
  is the \emph{spherical convolution} of $Q$ and $P$. Then the symbol 
  $\fun{K^{\wedge}}{k}$ is given by 
  $\fun{K^{\wedge}}{k} = \fun{Q^{\wedge}}{k} \fun{P^{\wedge}}{k}$.
  Furthermore, for compactly supported functions $\supp\; Q = \supp\; P =
  \interv{[}{h}{1}{]}$, $h>0$, we conclude $\supp\; K =
  \interv{[}{2h^2-1}{1}{]}$.
\end{lemma}

\section{Fast Summation}
Given a set of arbitrary \emph{source nodes} $\mathcal{Y} :=
  \pset{\V{\eta}_{l} \in \twosphere}{|}{l = 0,\ldots,L-1}$ and a vector of
  real coefficients $\V{b}:=(b_{l})_{l=0}^{L-1}$, our goal consists in the fast
evaluation of sums 
\begin{equation}
  \label{Applications:KernelSum}
  \fun{f}{\xi} := \sum_{l = 0}^{L-1} b_{l} \fun{K}{\V{\eta}_{l} \cdot \V{\xi}}
\end{equation}
on a set of arbitrary \emph{target nodes} $\mathcal{X} := \pset{\V{\xi}_{d}
  \in \twosphere}{|}{d=0,\ldots,D-1}$.

Given an explicit expression, the zonal function $\fun{K}{\V{\eta} \: \cdot}$ 
can be evaluated easily or all the values 
$\fun{K}{\V{\eta}_{l} \cdot \V{\xi}_{d}}$ can be stored in
advance. The naive approach evaluating \eqref{Applications:KernelSum} leads to
an $\bigo{L\:D}$ algorithm. 
For large $L$ and $D$ the computational effort becomes quickly unaffordable.
The \emph{panel clustering} method introduced in \cite{FrGlSch98} reduces the
computational effort to evaluate \eqref{Applications:KernelSum} based on the
traditional method of dividing the evaluation into near- and far-field.
For every zonal function $\fun{K}{\V{\eta}_l \: \cdot}$, the near-field contribution
is calculated exactly whereas the contribution of the far-field may be
approximated coarsly.
% due to the supposed rapid decay of $\fun{K}{\V{\eta} \:\cdot}$. 

We propose to simply truncate the series \eqref{Basics:Kernel} at degree 
$M \in \NZ$, i.e.
\begin{equation}
  \label{Applications:TruncatedSeries}
  \fun{K}{\V{\eta}_{l} \cdot \V{\xi}} \approx \fun{K_{M}}{\V{\eta}_{l} \cdot
  \V{\xi}} := \sum_{k=0}^{M} \sum_{n=-k}^k \fun{K^{\wedge}}{k}
  \fun{Y_{k}^n}{\V{\xi}} \overline{\fun{Y_{k}^n}{\V{\eta}_{l}}}.
\end{equation}

Substituting \eqref{Applications:TruncatedSeries} into
\eqref{Applications:KernelSum} and interchanging the order of summation we
finally obtain the approximation
\[
  \fun{f_{M}}{\xi_{d}} := \sum_{k=0}^{M} \sum_{n=-k}^k \fun{K^{\wedge}}{k}
  \paren{\sum_{l = 0}^{L-1} b_{l} \overline{\fun{Y_{k}^n}{\V{\eta}_{l}}}}
  \fun{Y_{k}^n}{\V{\xi}_{d}}.
\]

The expression in the inner brackets can be computed by an adjoint Nonuniform
Fast Spherical Fourier Transform (adjoint NFSFT) in 
$\mathcal{O}(L + M^2 \log^2 M)$
arithmetic operations, see \cite{} and the forthcoming paper \cite{} for
details.
This is followed by $(M+1)^2$ multiplications with the symbol
$\fun{K^{\wedge}}{k}$, and completed by a NFSFT to compute the outer sum in
$\mathcal{O}(D + M^2 \log^2 M)$ arithmetic operations.
In Section \ref{Basics:SphericalKernels}, we will decompose the error and show
that the degree $M$ depends only on the desired accuracy of our algorithm and
on the particular zonal function $\fun{K}{\V{\eta} \: \cdot}$, but not on the
numbers $L$ and $D$.
Thus, the overall arithmetic complexity of our algorithm is $\mathcal{O}(L +
D)$, in particular this performance does not depend on the distribution of the
nodes $\V{\xi}_{d}$ and $\V{\eta}_{l}$.

\begin{remark}
  NFSFT ist auch nur approximativ. Referenzen zu Driscoll, Healy, usw..... 
  (Aufbau des Algorithmus)
\end{remark}

\begin{remark}
In matrix-vector notation the proposed approach, a particular rank $(M+1)^2$
approximation, read as
\[
  \V{f} = \V{Y_{\mathcal{X}}} \: \V{\hat W} \:
  \V{Y_{\mathcal{Y}}}^{\adj} \: \V{b}
\]
with
\begin{align}
  \nonumber
  \V{f} & := \paren{\fun{f}{\V{\xi}_{d}}}_{d=0}^{D-1} \in \R^D,
  \\ \nonumber
  \V{Y_{\mathcal{X}}} & := \paren{\fun{Y_k^n}{\V{\xi}_{d}}}_{d=0,\ldots,D-1;
  k=0,\ldots,M,\:n=-k,\ldots,k} \in \C^{D \times
  \paren{M+1}^2}, \\ \nonumber
  \V{\hat W} & := \fun{\diag}{\V{\hat w}},\ \V{\hat w} := \paren{\hat
  w_{k}^{n}}_{k=0,\ldots,M,\:n=-k,\ldots,k} \in \R^{(M+1)^2},\ \hat w_{k}^n :=
  \fun{K^{\wedge}}{k}, \\ \nonumber
  \V{Y_{\mathcal{Y}}} & := \paren{\fun{Y_k^n}{\V{\eta}_{l}}}_{l=0,\ldots,L-1;
  k=0,\ldots,M,\:n=-k,\ldots,k} \in \C^{L \times \paren{M+1}^2}.
\end{align}

Replacing the NFSFT by its slow version, which takes $\mathcal{O}(L M^2)$ and
$\mathcal{O}(D M^2)$ arithmetic operations, respectively, yields an
$\mathcal{O}(L+D)$ algorithm, too.
\end{remark}

In summary, we propose the following algorithm.
\begin{algorithm}[h]
  \caption{Fast Summation}
  \label{Applications:Algorithm:FastSummation}    
  \begin{algorithmic}
    \STATE  Input:  $L \in \N$, $\paren{b_{l}}_{l=0}^{L-1}$, $\paren{\V{\eta}_{l}}_{l=0}^{L-1}$, 
                    $D \in \N$, $\paren{\V{\xi}_{d}}_{d=0}^{D-1}$, $M \in \NZ, \paren{\fun{K^{\wedge}}{k}}_{k=0}^M$
    \STATE
    \STATE Compute $\V{\tilde{b}} := \V{Y_{\mathcal{Y}}}^{\adj} \: \V{b}$ by an
                    adjoint NFSFT 
    \FOR {$k=0,\ldots,M$} 
      \FOR {$n=-k,\ldots,k$} 
        \STATE Set $a_{k}^n := \tilde{b}_{k}^n \: \fun{K^{\wedge}}{k}$
      \ENDFOR
    \ENDFOR
    \STATE Compute $\V{f} := \V{Y_{\mathcal{X}}} \: \V{a}$ by a NFSFT
    \STATE
    \STATE Output: $\paren{\fun{f}{\V{\xi}_{d}}}_{d=0}^{D-1}$ (an approximation to 
      $\paren{\fun{f}{\V{\xi}_{d}}}_{d=0}^{D-1}$)
    \STATE
    \STATE Complexity: $\mathcal{O}\left(M^2 \log^2M + L + D\right)$  
\end{algorithmic}
\end{algorithm}


%--------------------------------------------------------------------------
\section{Error estimates and examples}\label{Basics:SphericalKernels}
%--------------------------------------------------------------------------
\begin{lemma}\label{lemma:error}
  The proposed approximation obeys the uniform error estimate
  \begin{equation}
    \left\|f - f_{M}\right\|_{\infty} \le
    \left\|\V{b}\right\|_1 \left\|\fun{\left(K-K_M\right)}{\V{\eta} \: \cdot
      }\right\|_{\infty} \le \left\|\V{b}\right\|_1 \sum_{k>M}
    \frac{2k+1}{4\pi} \abs{\fun{K^{\wedge}}{k}}.
  \end{equation}
\end{lemma}

\subsection{Poisson and Singularity kernel}
\begin{definition}
  Let $h \in \interv{(}{0}{1}{)}$, then the
  \begin{enumerate}
  \item \emph{Poisson kernel}
    $Q_{h}:\interv{[}{-1}{1}{]} \rightarrow \R$ is given by
    \[
    \fun{Q_{h}}{x} := \frac{1}{4\pi} \frac{1-h^2}{\paren{1-2hx+h^2}^{3/2}}\,.
    \]
  \item and the \emph{singularity kernel}
    $S_{h}:\interv{[}{-1}{1}{]} \rightarrow \R$ is given by
    \[
    \fun{S_{h}}{x} := \frac{1}{2\pi} \frac{1}{\paren{1-2hx+h^2}^{1/2}},
    \]
  \end{enumerate}
\end{definition}

We refer to Figure \ref{Basics:Figure:PoissonSingularityKernel} for a visual
impression and mention that the parameter $h$ allows for controlling the
concentration of $\fun{Q_{h}}{\V{\eta} \: \cdot}$ and $\fun{S_{h}}{\V{\eta} \:
  \cdot}$ around $\eta \in \twosphere$, respectively.
The Poisson kernel is a positive function and normalized with
$\|\fun{Q_{h}}{\V{\eta} \: \cdot}\|_{\Ln{1}{\twosphere}}$.
Further properties with respect to localization and smoothness are derived
in \cite[pp. 112]{frgesc}.

\begin{figure}[tb]
  \centering
  \subfigure[$h=0.5,0.7,0.8$]
  {\includegraphics[width=0.33\textwidth]{images/poisson}}\hfill
  \subfigure[$h=0.8,0.9,0.95$]
  {\includegraphics[width=0.33\textwidth]{images/singularity}}
  \caption{The kernels $\fun{Q_{h}}{\cos\theta}$ and $\fun{S_{h}}{\cos\theta}$
  for different values of $h$.}
  \label{Basics:Figure:PoissonSingularityKernel}
\end{figure}

For both kernels the symbol is explicitely known, such that we easily state
the following lemma.
\begin{lemma}
 Using these kernels within our summation algorithm yields an relative error
 of
\begin{equation}
    \label{error:poisson}
    \frac{\left\|f - f_{M}\right\|_{\infty}}{\left\|\V{b}\right\|_1} \le
    \frac{h^{M+1}}{4\pi} \left(\frac{2M+1}{1-h}+\frac{2}{\left(1-h\right)^2}\right)
  \end{equation}
and
\begin{equation}
    \label{error:singular}
    \frac{\left\|f - f_{M}\right\|_{\infty}}{\left\|\V{b}\right\|_1} \le
    \frac{h^{M+1}}{4\pi} \left(\frac{2M+1}{2\left(1-h\right)}+
      \frac{4M}{\left(1-h\right)^2}+ \frac{4}{\left(1-h\right)^3}\right),
  \end{equation}
respectively.
\end{lemma}
\begin{proof}
Its symbol is given by $\fun{Q_{h}^{\wedge}}{k}=h^k$.
   with symbol $\fun{S_{h}^{\wedge}}{k} = \frac{2k+1}{2} h^k$.
\end{proof}
  
Simply put, our scheme (almost) achieves accuracy $\epsilon$ for $M \ge \log\epsilon \,
/ \, \log h$.

\subsection{Locally supported kernel}
  Let $h \in \interv{(}{0}{1}{)}$ and $\lambda \in \NZ$.
  A locally supported zonal function $L_{h,\lambda}:\interv{[}{-1}{1}{]}
  \rightarrow \R$, considered in \cite{Sc97}, is defined by
  \[
  \fun{L_{h,\lambda}}{x} := 
  \left\{\begin{array}{l@{\quad \text{if} \quad}rcl} 
      0 & -1 &\le x \le& h, \\
      \frac{\lambda+1}{2\pi(1-h)^{\lambda+1}}\paren{x-h}^{\lambda} &  h & <  x \le& 1,
    \end{array}\right.
  \]
  has the symbol $\fun{L_{h,\lambda}^{\wedge}}{k}$, which can be computed
  recursively, using \eqref{three1}, by
  \[
  \fun{L_{h,\lambda}^{\wedge}}{k+1} = \frac{\paren{2k+1} h}{k+\lambda+2}
  \fun{L_{h,\lambda}^{\wedge}}{k}   - \frac{k-\lambda-1}{k+\lambda+2}
  \fun{L_{h,\lambda}^{\wedge}}{k-1}
  \]
  for $k\in \N$, $\fun{L_{h,\lambda}^{\wedge}}{0} = 1$, and
  $\fun{L_{h,\lambda}^{\wedge}}{1} = \frac{\lambda + 1 + h}{\lambda+1}$.
  Figure \ref{Basics:Figure:LKernel} shows the function $L_{h,\lambda}$ for
  different values $h$ and $\lambda$.
  \begin{figure}[tb]
    \centering
    \subfigure[$\lambda=1.5$]
    {\includegraphics[width=0.33\textwidth]{images/locsup4}}\hfill
    \subfigure[$\lambda=1.0$]
    {\includegraphics[width=0.33\textwidth]{images/locsup3}}\\
    \subfigure[$\lambda=0.5$]
    {\includegraphics[width=0.33\textwidth]{images/locsup2}}\hfill
    \subfigure[$\lambda=0.2$]
    {\includegraphics[width=0.33\textwidth]{images/locsup1}}
    \caption{The kernel $L_{h,\lambda}$ for $h = -0.7, -0.2, 0.2, 0.7$ and different values of $\lambda$.}
    \label{Basics:Figure:LKernel}
  \end{figure}
  While the parameter $h$ again steers the localisation in spatial domain, the
  parameter $\lambda$ correponds to the localisation in frequency domain.
  Using in particular \eqref{three2}, we obtain for $\lambda=0$ the decay
  \[
  \left|\fun{L_{h,0}^{\wedge}}{k}\right| = \frac{1}{1-h}
  \left|\int_{h}^{1} \fun{P_{k}}{x} \dx x\right| \le
  \frac{\sqrt{2}}{\left(2k+1\right)\left(1-h\right)\sqrt{\pi}\sqrt[4]{1-h^2}}
  \left(\frac{1}{\sqrt{k-1}}+\frac{1}{\sqrt{k+1}}\right)\,.
  \]
  Unfortunately, this is not sufficient for an estimate of the relative error
  by means of Lemma \ref{lemma:error}.
  Nevertheless, using a technique we estimate
  \[
  \left|\fun{L_{h,\lambda}^{\wedge}}{k}\right| \le C_h
  k^{-\frac{3}{2}-\lambda}\,.
  \] 

\subsection{Gaussian kernel}
  Let $\sigma>0$ and the \emph{Gaussian kernel}
  $G_{\sigma}:\interv{[}{-1}{1}{]} \rightarrow \R$ be given by
  \begin{equation}
    \label{GaussKernel}
    \nonumber
    \fun{G_{\sigma}}{x} := \e^{2\sigma x-2\sigma}\,.
  \end{equation}
  Again, the symbol $\fun{G_{\sigma}^{\wedge}}{k}$ can be computed by a
  recurrence relation
  \[
  \fun{G_{\sigma}^{\wedge}}{k+1} = ?
  \fun{G_{\sigma}^{\wedge}}{k}   - ?
  \fun{G_{\sigma}^{\wedge}}{k-1}
  \]
  for $k\in \N$, $\fun{G_{\sigma}^{\wedge}}{0} = 4 \pi \sigma^{-1}
  \e^{-2\sigma} \sinh \sigma \cosh \sigma$, and $\fun{G_{\sigma}^{\wedge}}{1} = \pi \sigma^{-2}
  \e^{-2\sigma} (2 \sigma \cosh 2 \sigma + \sinh \sigma )$.

%--------------------------------------------------------------------------
\section{Numerical experiments}
%--------------------------------------------------------------------------

We present numerical experiments in order to demonstrate the performance of
our algorithm.
All algorithms were implemented in C and tested on an AMD Athlon\texttrademark
XP 2700+ with 2GB main memory, SuSe-Linux (kernel 2.4.20-4GB-athlon, gcc 3.3)
using double precision arithmetic. 
Moreover, we have used the libraries FFTW 3.0.1 \cite{fftw} and NFFT 2
\cite{kupo02C}. 
Throughout our experiments we have applied the NFFT package \cite{kupo02C}
with precomputed Kaiser--Bessel functions and an oversampling factor $\rho=2$.

In our tests we have always chosen random source and target {\bf wie gewählt}
and coefficients $b_l$ uniformly distributed in the complex box
$[-\frac{1}{2},\frac{1}{2}]\times [-\frac{1}{2},\frac{1}{2}]{\rm i}$.

We have considered the following kernels

\begin{itemize}
\item ...,
\end{itemize}


First we examine the errors that are generated by our approach.
Figure \ref{fig:error} presents the error
\[
E_{\infty}:=
 \frac{\max\limits_{d=0,\ldots,D-1}\left|\fun{f}{\xi_{d}}-
 \fun{f_{M}}{\xi_{d}}\right|}{\sum\limits_{k=0}^N 
 |b_l|} \quad \approx \quad \|K_{\text{ERR}}\|_{\infty}
\]
introduced by our algorithms as function of the parameter $M$.

These results confirm the error estimates in XXX

\begin{figure}[ht]
  \centering
  \begin{tabular}[h]{cc}
    XX & XX
  \end{tabular}
  \caption[fig:Time]{Error $E_{\infty}$ for $M=8,12,16,\hdots,128$ and $D=L=1000$.
    \label{fig:error}}
\end{figure}

Finally, we compare the computation time of the straightforward summation, the
straightforward summation with the precomputed matrix, the fast summation
algorithm with NDSFT, and the fast summation algorithm with NFSFT for
increasing $D=L$. 
The CPU time required by the four algorithms is shown in Table
\ref{tab:TimeSpace}. 
As expected the fast summation algorithm outperform the straightforward
algorithms, yielding an ${\cal O}(D)$ complexity in both variants, whereas the
NFSFT--version is considerably faster.



\begin{table}[ht!]
  \begin{center}
    \begin{tabular}{r|r|r|r|r|r}
      $D=L$    & direct alg.   & with precomp. & fast GT, NDFT & fast GT, NFFT & error $E_\infty$ \\ \hline
    \end{tabular}
  \end{center}
  \caption{CPU-Time and error $E_{\infty}$ for the fast summation algorithm.
    Note that we used accumulated measurements in case of small times and the
    times/error (*) are not displayed due to the large response time or the 
    limited size of memory.}
  \label{tab:TimeSpace}
\end{table}


%--------------------------------------------------------------------------
\section{Conclusions}
%--------------------------------------------------------------------------

We have presented a fast algorithm for the computation of sums of type
\eqref{Applications:KernelSum} in ${\cal O} (D +L)$ arithmetic operations.
We have proved error estimates concerning the dependence of the computational
speed on the desired accuracy and the parameters of particular interesting
kernels.
The software for this algorithm including all described tests is available
within the NFFT package \cite[{\tt ./example/fastsumS2}]{kupo02C}. // Can be
obtained from the authors?????

%-----------------------------------------------------------------------------
\bibliographystyle{abbrv}
%\bibliography{/home/potts/tex/ref}
%\bibliography{ref}
%\bibliography{../myrefs}
\bibliography{/home/kunis/Paper/ref}
\end{document}
