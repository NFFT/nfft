%=============================================================================
\documentclass[11pt,a4paper,twoside]{article}
%=============================================================================
\usepackage{amsfonts}
\usepackage{amsmath}
\usepackage{theorem}
%\usepackage{showkeys}
%\usepackage{psfig}
%\usepackage{lscape}
\usepackage{color}
\usepackage{graphics}
\usepackage{graphicx}
\usepackage[boxed,Algorithm]{algorithm}
\usepackage{algorithmic}

\parindent0mm

\begin{document}
\begin{center}
{\bf \large intern.tex}\\
\today
\end{center}

In the following float\_type is one of
\begin{itemize}
 \item double
 \item fftw\_complex
\end{itemize}

The abstract\_plan contains
\begin{itemize}
 \item int M\_total
 \item int N\_total
 \item float\_type *f
 \item float\_type *f\_hat
\end{itemize}
and provides the routines
\begin{itemize}
 \item void abstract\_trafo(abstract\_plan *this)
 \item void abstract\_adjoint(abstract\_plan *this)
\end{itemize}

The sub directories of nfft are
\begin{itemize}
 \item doc
 \item example
 \item nfct
 \item nfft
 \item nfst
 \item nfsft
 \item nnfft
 \item nsfft
 \item obsolete
 \item solver
 \item util
\end{itemize}

\end{document} 