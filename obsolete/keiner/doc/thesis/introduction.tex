\chapter{Introduction} \label{Introduction}

The \emph{fast Fourier transform (FFT)} has become one of the most important 
and widely used algorithms today. In 1965, Cooley and Tukey \cite{cotu}
published and publicized the first description of a fast algorithm, known
nowadays as \emph{Cooley-Tukey FFT}, for computing certain trigonometric sums
\emph{Fourier sums},
generally referred to as the  \emph{discrete} or \emph{finite Fourier transform
(DFT)}. In fact, developments since then now let us speak of a whole class of
algorithms. The original Cooley-Tukey FFT used a recursive scheme to split one
large transform of length $N \in \N$ successively into smaller transforms 
achieving an asymptotical
complexity of  $\bigo{N \log N}$ \emph{floating point operations (flops)}
instead of $\bigo{N^2}$ for a direct computation.  Efficient and highly 
optimized implementations also for the
multidimensional case are available \cite{fftw}. Surprisingly, the idea of
Cooley and Tukey was already described by Gauss around 1805 \cite{HeiJoBu85}
interpolating the trajectories of the asteroids Pallas  and Juno. But his work
was not widely recognized and only published posthumously. The wide areas of
applications today using FFT-techniques, including for example, 
time-frequency analysis, signal-processing and the numerical solution of 
partial differential equations, underline the great importance and impact of
FFT algorithms.

Recently, algorithms for a more general type of discrete Fourier
transform, namely the \emph{nonuniform discrete Fourier transform (NDFT)},
have been developed. 
While the DFT is a bijective and easily invertible linear mapping of $N$ 
ingoing coefficients to $N$ outgoing coefficients, corresponding to uniformly
distributed nodes on a Torus, the NDFT generalizes Fourier sums for
arbitrary node distributions, hence the name 'nonuniform'. Fast algorithms have
been described in several papers  (see \cite{postta01} and the references
therein) and a C subroutine library is available \cite{kupo02C}.

A different line of generalization leaves the 'classical' setting of  Fourier
series on a multidimensional torus and analogously describes Fourier series on
different geometries like the surface of a multidimensional sphere.
Particularly, the two-dimensional unit sphere $\twosphere$ embedded 
into $\R^3$ has practical
relevance in quite a wide range of  applications, in many cases owing to its
correspondence to the surface of the earth. Fields of interest are numerical
analysis in geo-sciences, for example in the computation of climate models for
weather forecasts or regarding the gravitational potential of the  earth.
Moreover, data distributed on the surface of a sphere arises in many different
applications in a natural way. Examples are molecular dynamics for protein
docking problems and  gamma-cameras for applications in computed tomography.
Unfortunately, the spherical setting differs from the 'classical'  one in that
the numerical treatment of many computational problems is quite more
challenging. Although there exists a rich theory for  analysis on the sphere,
fast algorithms allowing for efficient and reliable discrete transforms in
terms of so-called  \emph{spherical surface functions}, the spherical
counterpart of the  Fourier basis functions on the torus, had not been
developed until a first paper by Driscoll and Healy  \cite{drhe}.

During the past years, major progress has been made and several  different
techniques have been proposed (see \cite{HeRoKoMo},  \cite{suta}, \cite{roty}).
The basic idea for the  methods presented in this text is to convert the
discrete spherical  Fourier transform into a 'classical' two-dimensional
discrete  Fourier transform. A careful analysis of numerical instabilities
involved has proven to be of major  importance.

The aim of this text is to describe particular algorithms for fast discrete
spherical Fourier transforms. Chapter \ref{Basics} gives a brief  introduction
to Fourier analysis on the sphere $\twosphere$ and  provides fundamental facts
and results. We refer to the literature for  details.

In Chapter \ref{DSFT}, algorithms for the computation of discrete spherical
Fourier transforms for arbitrary nodes on the sphere  are presented. We first
describe the so-called 'direct' algorithms  and mention how spherical Fourier
coefficients are computed from function samples by means of quadrature formulae: Starting with a
simple algorithm  for the evaluation of a \emph{spherical Fourier sum} at
arbitrary  nodes, we derive the matrix factorization corresponding to the 
algorithm translated into matrix-vector notation. Based on the  easily obtained
\emph{adjoint matrix factorization}, we  describe the so-called \emph{adjoint
algorithm}. In conjunction with appropriate quadrature formulae, we are able to
reconstruct spherical Fourier coefficients from function samples on the sphere.

We then present 'fast' algorithms which employ a \emph{discrete
polynomial transform}. The work is based on  previous papers concerning an
algorithm for  the fast approximative evaluation of \emph{spherical Fourier
sums} at arbitrary nodes on the 
sphere. The main contribution of this text is the derivation of the
corresponding matrix factorization along with the construction of the 
\emph{adjoint algorithm}. All algorithms were implemented and tests  of the
'fast' algorithms with respect to numerical stability and  runtime performance
are provided. 

Chapter \ref{Applications} finally presents two applications of
the  proposed algorithms: 

We propose a new method for the fast  approximative summation of
\emph{spherical radial basis functions} and establish theoretical error bounds for
some example functions. The  theoretical results are verified by numerical
tests and the runtime performance of the presented algorithms is compared with
a direct and exact implementation.

We finally treat the generally ill-posed problem of reconstructing  spherical
Fourier coefficients from scattered data. We formulate the problem as a
\emph{weighted linear least squares problem} and a \emph{weighted optimal
interpolation problem} using \emph{normal equations}. We then describe how the fast
algorithms for discrete spherical Fourier transforms can be used to obtain  a
solution by means of standard iterative algorithms for solving systems of linear
equations. Examples are given for the former case using  a real-life dataset.

