\chapter{Nonequispaced Discrete Spherical Fourier Transform}
\label{DSFT}
Let in this chapter $M \in \NZ$ be fixed, $N := 2^t$ be the 
next greater power of two with respect to $M$, i.e. $t := {\ceil{\log_2 M}}$, and $\mathcal{X} := \paren{\vtheta_d,\vphi_d}_{d=0}^{D-1}$ for $D \in \N$ be
a set of nodes on $\twosphere$ called the \emph{sampling set}. As a special case, a \emph{spherical grid} 
$\mathcal{X}$ is defined by a Cartesian product 
$$
  \mathcal{X} := \pset{\vtheta_{l}}{|}{l=0,\ldots,L-1} \times \pset{\vphi_{j}}{|}{j=0,\ldots,J-1} \quad \paren{J,L \in \N}
$$
with colatitudes $\vtheta_{l} \in \interv{[}{0}{\pi}{]}$ and longitudes $\vphi_{j} \in \interv{[}{0}{2\pi}{)}$.

Given a function $f: \twosphere \rightarrow \C$, $f \in \fun{\Pol_{M}}{\twosphere}$ with Fourier expansion
\begin{equation}
  \label{NFSFT:FourierExpansion} 
  f = \sum_{\paren{k,n} \in\mathcal{I}^M} \fun{a_k^n}{f} Y_{k}^n = \sum_{k=0}^M \sum_{n=-k}^k \fun{a_k^n}{f} Y_{k}^n = \sum_{n=-M}^M \sum_{k=\abs{n}}^M \fun{a_k^n}{f} Y_{k}^n,
\end{equation}  
where 
$$\mathcal{I}^M := \pset{\paren{k,n}}{|}{k=0,\ldots,M;n=-k,\ldots,k},$$
the \emph{nonequispaced discrete spherical Fourier transform (NDSFT)} maps the coefficients $\paren{\fun{a_k^n}{f}}_{(k,n) \in \mathcal{I}^M}$ to the function values $\fun{f}{\mathcal{X}}$ on the sampling set $\mathcal{X}$.
We call $M$ the \emph{bandwidth} of $f$ and $a_{k}^n = \fun{a_{k}^n}{f}$ are the \emph{spherical Fourier coefficients} of $f$ with respect to the 
orthonormal basis of spherical harmonics $\set{Y_{k}^n}_{\paren{k,n} \in\mathcal{I}^M}$ of $\fun{\Pol_{M}}{\twosphere}$. The index $k$ will always denote the degree and $n$ always the order of $Y_{k}^n$. The Fourier coeffcients $a_{k}^n$ may be ordered 
in different ways: We refer to the \emph{degree-major order} when the Fourier coefficents are sorted first by degree $k$ and then by order $n$ in ascending order, hence
$$ a_{0}^0,\: a_{1}^{-1},\: a_{1}^{0},\: a_{1}^{1},\: a_{2}^{-2},\: \ldots,\: a_{M}^{M-1},\: a_{M}^{M}.$$ 
Analogously, the \emph{order-major order} corresponds to reversed precedence, i.e.
$$ a_{M}^{-M},\: a_{M-1}^{-(M-1)},\: a_{M}^{-(M-1)},\: a_{M-2}^{-(M-2)},\: \ldots,\: a_{M-1}^{M-1},\: a_{M}^{M-1},\: a_{M}^{M}.$$ 
From a linear algebra point of view, evaluating $f$ on $\mathcal{X}$ corresponds to a matrix-vector product
$$ \fun{\V{f}}{\mathcal{X}} = \fun{\V{Y}}{\mathcal{X}} \; \V{a}$$
with
\begin{equation}
  \nonumber
  \begin{split}
    \fun{\V{f}}{\mathcal{X}} & := \paren{f_d}_{d=0}^{D-1} \in \C^{D},\ f_{d} := \fun{f}{\vtheta_{d},\vphi_{d}},\\
    \fun{\V{Y}}{\mathcal{X}} & := \paren{\fun{Y_k^n}{\vtheta_d,\vphi_d}}_{d=0,\ldots,D-1; \paren{k,n} \in \mathcal{I}^M} \in \C^{D \times \paren{M+1}^2},\\
    \V{a} & := \paren{a_k^n}_{\paren{k,n} \in \mathcal{I}^M} \in \C^{\paren{M+1}^2}.
  \end{split}
\end{equation}
If not specified otherwise, $\V{a}$ contains the Fourier coeffcients $a_{k}^n$ in degree-mayor order and the columns of $\fun{\V{Y}}{\mathcal{X}}$ 
are ordered accordingly. 
%We drop the indices $\mathcal{X}$ and $M$ where appropriate, i.e. where they could be chosen arbitrarily or are obvious from the 
%context and write $\V{f} = \V{Y} \; \V{a}$ instead. 
Furthermore, we introduce the denotation
\begin{align*}
    \V{a}_{k} & := \paren{a_{k}^{n}}_{n=-k,\ldots,k} \in \C^{2k+1} & \paren{k=0,\ldots,M},\\
    \V{a}^{n} & := \paren{a_{k}^{n}}_{k=\abs{n},\ldots,M} \in \C^{M-\abs{n}+1} & \paren{n=-M,\ldots,M}
\end{align*} 
for subvectors of $\V{a}$. If $\mathcal{X}$ is a spherical grid, the standard order is first by colatitude $\vtheta_{l}$ and then by longitude $\vphi_{j}$. 
We use double subscripts $l,j$ to denote $\V{f}_{l,j} := \fun{f}{\vtheta_{l},\vphi_{j}}$.

A first "naive" approach to evaluating $f$ on $\mathcal{X}$ according to \eqref{NFSFT:FourierExpansion} would be to compute first all $D\:(M+1)^2$
entries of $\V{Y}$ and then perform the matrix-vector multiplication $\fun{\V{Y}}{\mathcal{X}} \: \V{a}$ the standard way. This would lead to an $\bigo{M^3 \: D}$
algorithm, since the evaluation of a function $Y_{k}^n$ at a single node already requires $\bigo{M}$ \emph{floating-point operations (flops)}.
But one can do better. Algorithm \ref{NFSFT:directDSFT} referred to as \emph{direct NDSFT} computes 
for every $\paren{\vtheta_{d},\vphi_{d}} \in \mathcal{X}$ first the sums 
\begin{equation}
  \label{NFSFT:directDSFT:firstSum}
  b^{n}_{d} := \sum_{k=\abs{n}}^M a_{k}^n \fun{P_{k}^{\abs{n}}}{\cos\vtheta_{d}} \quad \paren{n=-M,\ldots,M}. 
\end{equation}
Here one can employ the \emph{Clenshaw algorithm} (see for example \cite{prtevefl}) using the three-term recurrence 
for the associated Legendre functions from \eqref{Basics:AssociatedLegendreDefinition}. We finally compute
\begin{equation}
  \label{NFSFT:directNDSFT:secondSum}
  \fun{f}{\vtheta_{d},\vphi_{d}} = \sum_{n=-M}^M b^{n}_d \e^{\im n \vphi_{d}}
\end{equation}
directly.
\begin{algorithm}[b]
  \caption{Direct DSFT}
  \label{NFSFT:directDSFT}    
  \begin{algorithmic}
    \STATE  Input: $M \in \NZ$, $D \in \N$, $\V{a}$, $\mathcal{X}$ %$\paren{a_{k}^n}_{\paren{k,n} \in I^M}$
    \STATE
    \FOR {$d=0,\ldots,D-1$} 
      \STATE $f_{d} := 0$
      \FOR {$n=-M,\ldots,M$} 
        \STATE Compute $\fun{b^{n}}{d}$ by the Clenshaw algorithm,
        \STATE $f_{d} := f_{d} + b^n_d \e^{\im n \vphi_{d}}$
      \ENDFOR
    \ENDFOR
    \STATE
    \STATE Output: $\paren{f_{d}}_{d=0,\dots,D-1}$
\end{algorithmic}
\end{algorithm}
Each application of the Clenshaw algorithm for \eqref{NFSFT:directDSFT:firstSum} needs $\bigo{M}$ 
flops summing up to $\bigo{M^2}$ flops for the complete inner loop. In 
total we have an $\bigo{M^2\:D}$ algorithm for the evaluation of $f$ on $\mathcal{X}$ and gained an order of complexity with respect to $M$. 
For special sampling sets one can exploit the grid 
structure and employ FFT techniques for the sums \eqref{NFSFT:directNDSFT:secondSum} if the nodes $$\varphi_{d}$$  are equispaced 
(see \citelist{\cite{drhe}} \cite{postta97} \cite{kupo02}). 

%\citelist{\cite{Mo99} \cite{postta97} \cite{roty} \cite{suta}}

\section{Computing Fourier Coeffcients from Function Samples}
A \emph{discrete Fourier transform (DFT)} on the torus $\mathbb{T} := \pset{x \in \R}{|}{0 \le x < 2\pi}$ of length $M \in \NZ$ can be represented 
as a maxtrix-vector product
$$
  \V{f} = \V{F} \: \V{\hat{f}}
$$
with
$$
 \V{F} := \paren{\e^{2 \pi i j \frac{k}{M}}}_{j,k = 0}^{M-1} \in \C^{M \times M},\ \V{\hat{f}},\:\V{f} \in \C^{M},
$$
as well. The Fourier matrix $\V{F}$ is unitary yielding the inversion formula
\begin{equation}
  \label{NFSFT:DFTInversion}
  \V{\hat{f}} = \V{F}^{\h} \: \V{f}
\end{equation}
immediately. The DFT evaluates a trigonometric polynomial 
$$
  f := \sum_{k=0}^{M-1} \fun{\hat{f}}{k} \e^{2 \pi \im k x}
$$
at equidistant nodes $\paren{x_{d}}_{d=0}^{M-1}$ with $x_{d} := \frac{d}{M} \in \mathbb{T}$. 
For an arbitrary sampling set $\mathcal{Y}$ the associated Fourier matrix $\fun{\V{F}}{\mathcal{Y}}$ is in general no longer unitary. 
Depending on the number of Fourier coefficients and the number of nodes,
$\fun{\V{F}}{\mathcal{Y}}$ is moreover no longer square.

Having defined the NDSFT as the evaluation of a bandlimited function $f$ on a sampling set $\mathcal{X}$ on $\twosphere$, arises the question 
whether there exist sets $\mathcal{X}$ allowing for an inversion formula similar to \eqref{NFSFT:DFTInversion}.
The answer is affirmative but in general the number of nodes needed exceeds the number of Fourier coefficients to be computed, 
i.e. $\fun{\V{Y}}{\mathcal{X}}$ is not square.

We want to recover the Fourier coefficients $a_{k^n}$ from samples $\fun{\V{f}}{\mathcal{X}}$ at certain nodes $\mathcal{X}$. They 
are given by the scalar products
$$
  a_{k}^n = \scalarproduct{f}{Y_{k}^n}_{\twosphere} = \int_{0}^{2\pi} \int_{0}^{\pi} \fun{f}{\vtheta,\vphi} \overline{\fun{Y_{k}^n}{\vtheta,\vphi}} \sin \vtheta \; \dx \vtheta \; \dx \vphi \quad \paren{\paren{k,n} \in \mathcal{I}^M}.
$$
By seperating the integrand with respect to the integration variables, we obtain
\begin{eqnarray*}
  a_{k}^n & = & \int_{0}^{2\pi} \int_{0}^{\pi} \fun{f}{\vtheta,\vphi} \fun{P_{k}^{\abs{n}}}{\cos \vtheta} \e^{-\im n \vphi} \sin \vtheta \; \dx \vtheta \; \dx \vphi\\
          & = & \int_{0}^{\pi} \fun{P_{k}^{\abs{n}}}{\cos \vtheta} \sin \vtheta \; \int_{0}^{2\pi} \fun{f}{\vtheta,\vphi} \e^{-\im n \vphi} \; \dx \vphi \; \dx \vtheta.
\end{eqnarray*}
For the integrals
$$
  \fun{f_{n}}{\vtheta} := \int_{0}^{2\pi} \fun{f}{\vtheta,\vphi} \e^{-\im n \vphi} \; \dx \vphi
$$
we have the quadrature rule
$$ \fun{f_{n}}{\vtheta} = \frac{1}{2(M+1)} \sum_{j=0}^{2M+1} \fun{f}{\vtheta,\vphi_{j}} \e^{-\im n \vphi_{j}}$$
where
$$ \vphi_{j} := \frac{j\pi}{N} \quad \paren{j=0,\ldots,2M+1}. $$
This holds due to the Sampling Theorem by taking into account that for fixed $\vtheta$ the function
\begin{eqnarray*}
  \fun{f}{\vtheta,\vphi} & = & \sum_{n=-M}^{M} \left(\sum_{k=\abs{n}}^M a_{k}^n \fun{P_{k}^{\abs{n}}}{\cos \vtheta}\right) \e^{-\im n \vphi}\\
\end{eqnarray*}
is a trigonometric polynomial of degree $M$ in $\vphi$ with Fourier coefficients
$$
  c_{n} := \sum_{k=\abs{n}}^M a_{k}^n \fun{P_{k}^{\abs{n}}}{\cos \vtheta}.
$$
It remains to compute
$$
  a_{k}^n = \int_{0}^{\pi} \fun{P_{k}^{\abs{n}}}{\cos \vtheta} \fun{f_{n}}{\vtheta} \sin \vtheta \; \dx \vtheta = 
  \int_{-1}^1 \fun{P_{k}^{\abs{n}}}{x} \fun{f_{n}}{\arccos x} \; \dx x.
$$
By verifying that $\fun{P_{k}^{\abs{n}}}{x} \fun{f_{n}}{\arccos x}$ is an algebraic polynomial of degree at most $2M+1$, 
we can use various types of quadrature rules. We first mention the \emph{Gauss-Legendre} quadrature rule
$$
  \int_{0}^{\pi} \fun{P_{k}^{\abs{n}}}{\cos \vtheta} \fun{f_{n}}{\vtheta} \sin \vtheta \; \dx \vtheta = \sum_{l=0}^{M} w_{l}^{\gl} \fun{P_{k}^{\abs{n}}}{\cos \vtheta_{l}^{\gl}} \fun{f_{n}}{\vtheta_{l}} 
$$
with nodes $\paren{\vtheta_{l}^{\gl}}_{l=0}^M$ and weights $\paren{w_{l}^{\gl}}_{l=0}^M$ as described for example in \cite{boehme02}. 
We notice that the $M+1$ nodes $\paren{\vtheta_{l}^{\gl}}_{l=0}^{M}$ can be computed as the eigenvalues of the \emph{Jacobi matrix} for the orthogonal 
Legendre polynomials. The weights $\paren{w_{l}}_{l=0}^{M}$ are obtained from the corresponding eigenvectors.

A second idea is to employ a Clenshaw-Curtis quadrature rule (see \cite{dara}, pp. 86)
$$
  \int_{0}^{\pi} \fun{P_{k}^{\abs{n}}}{\cos \vtheta} \fun{f_{n}}{\vtheta} \sin \vtheta \; \dx \vtheta = \sum_{l=0}^{2M} \varepsilon_{l}^{2M} w_{l}^{\cc} \fun{P_{k}^{\abs{n}}}{\cos \vtheta_{l}^{\cc}} \fun{f_{n}}{\vtheta_{l}}
$$
where $\varepsilon_{0}^{M} := \epsilon_{M}^M := \frac{1}{2}$, $\epsilon_{l}^M := 1$, $l=1,\dots,M-1$, and $\vtheta_{l}^{\cc} = \frac{l\pi}{2M}$ are the \emph{Chebyshev nodes}.
This quadrature rule uses almost twice as many points as the Gauss-Legendre quadrature rule but allows for an easy and fast online computation of the nodes $\vtheta_{l}^{\cc}$
and weights $w_{l}^{\gl}$. The weights are given by
$$ w_{l}^{\gl} := \frac{1}{2M} \sum_{j=0}^{M} \varepsilon_{j}^{M} \frac{-2}{4j^2-1} \cos\frac{lj\pi}{M}.$$
This finally yields for the Fourier coeffcients $a_{k}^n$ the identities 
\begin{equation}
  \label{NFSFT:iDSFT}
  \begin{split}
    a_{k}^n & = \frac{1}{2(M+1)} \sum_{j=0}^{2(M+1)-1} \sum_{l=0}^{M} w_{l}^{\gl} \fun{f}{\vtheta_{l}^{\gl},\vphi_{j}} \overline{\fun{Y_{k}^n}{\vtheta_{l}^{\gl},\vphi_{j}}},\\
    a_{k}^n & = \frac{1}{2(M+1)} \sum_{j=0}^{2(M+1)-1} \sum_{l=0}^{2M} \varepsilon_{l}^{2M} w_{l}^{\cc} \fun{f}{\vtheta_{l}^{\cc},\vphi_{j}} 
  \overline{\fun{Y_{k}^n}{\vtheta_{l}^{\cc},\vphi_{j}}}.
  \end{split}
\end{equation}
Associated with this quadrature formulae, we define the sampling sets
\begin{eqnarray*}
  \mathcal{X}^{\gl} & := & \paren{\vtheta_{l}^{\gl},\vphi_{j}}_{j=0,\dots,2(M+1)-1;l=0,\dots,M},\\
  \mathcal{X}^{\cc} & := & \paren{\vtheta_{l}^{\cc},\vphi_{j}}_{j=0,\dots,2(M+1)-1;l=0,\dots,2M}
\end{eqnarray*}
and refer to $\mathcal{X}^{\gl}$ as the \emph{Gauss-Legendre sampling set of degree M} and to $\mathcal{X}^{\cc}$ as the 
\emph{Clenshaw-Curtis sampling set of degree M}. These two types of sampling sets are illustrated in Figure \ref{quadrature}.
\begin{figure}[tb]
  \centering
  \includegraphics[width=12cm]{images/quadrature}
  \caption{The sample sets $\mathcal{X}^{\gl}$ (red dots) and $\mathcal{X}^{\cc}$ (green dots) for $M=7$.}
  \label{quadrature}
\end{figure}
We now immediately obtain the corresponding inversion formulae
\begin{equation}
  \label{NFSFT:InversionFormula}
  \begin{split}
    \V{a} & = \fun{\V{Y}}{\mathcal{X}^{\gl}}^{\h} \: \V{W}^{\gl} \: \fun{\V{f}}{\mathcal{X}^{\gl}},\quad \V{W}^{\gl} := \V{I}_{2M+2} \otimes \diag\paren{w_{l}^{\gl}}_{l=0}^M,\\
    %\label{NFSFT:InversionFormulaCC}
    \V{a} & = \fun{\V{Y}}{\mathcal{X}^{\cc}}^{\h} \: \V{W}^{\cc} \: \fun{\V{f}}{\mathcal{X}^{\cc}},\quad \V{W}^{\cc} := \V{I}_{2M+2} \otimes \diag\paren{\varepsilon_{l}^{2M}w_{l}^{\cc}}_{l=0}^M.
  \end{split}  
\end{equation}  
An algorithm for the multiplication with the adjoint matrix $\V{Y}^{\h}$ is the key for an implementation according to \eqref{NFSFT:InversionFormula}. Evaluating the matrix-vector product the standard way, i.e. row-wise for $\V{Y}^{\h}$, a first approach would be to compute the Fourier coefficients $a_{k}^n$ by \eqref{NFSFT:iDSFT} for all $\paren{k,n} \in \mathcal{I}^M$ employing the Clenshaw algorithm for the evaluation of the functions $Y_{k}^n$. But this would again lead to an $\bigo{M^3 \: D}$ algorithm. This is due to the fact, that the Clenshaw algorithm would now be applied to each function $Y_{k}^n$ seperately instead of evaluating a linear combination for fixed $n$ and $k = \abs{n},\dots,M$ as in Algorithm \ref{NFSFT:directDSFT}. If we recall that we would end up with a $\bigo{M^3 \: D}$ algorithm for the NDSFT, too, if we evaluated the matrix $\V{Y}$ componentwise and then performed the matrix-vector multiplication as usual, we notice that Algorithm \ref{NFSFT:directDSFT} implies a factorization of $\V{Y}$ into a product of sparse matrices therefore allowing us to save one order of $M$. Furthermore, by taking the adjoint factorization, we obtain a corresponding factorization of $\V{Y}^{\h}$ immediately. This allows for the derivation of an algorithm of exactly the same asymptotic complexity for computing the adjoint product, too.

\section{A Factorization of the Fourier Matrix}

In order to abtain an $\bigo{M^2 \: D}$ algorithm for the adjoint product, we first derive the factorization of $\V{Y}$ according to Algorithm \ref{NFSFT:directDSFT}. The matrix $\V{Y}$ can be written as
$$
  \V{Y} = 
    \left[\begin{array}{c}
      \V{E}_{0}\\
      \V{E}_{1}\\
      \vdots\\
      \V{E}_{D-1}
    \end{array}\right], \quad \V{E}_{d} \in \C^{1 \times (M+1)^2} \quad \paren{d=0,\ldots,D-1}.
$$
Each matrix $\V{E}_{d}$ evaluates $f$ at $\paren{\vtheta_{d},\vphi_{d}}$ according to 
$$
  \fun{f}{\vtheta_{d},\vphi_{d}} = \sum_{n=-M}^M \sum_{k=\abs{n}}^M a_k^n \fun{Y_{k}^n}{\vtheta_{d},\vphi_{d}}.
$$
So let $d$ be fixed. The algorithms begins by evaluating the sums
$$
  b^{n}_{d} = \sum_{k=\abs{n}}^M a_{k}^n \fun{P_{k}^{\abs{n}}}{\cos\vtheta_{d}} \quad \paren{n=-M,\ldots,M}. 
$$
which corresponds to computing
$$
  \left(\begin{array}{c}
    b^{-M}_{d}\\
    b^{-(M-1)}_{d}\\
    \vdots\\
    b^{M}_{d}
  \end{array}\right)
  =
  \left[\begin{array}{cccc}
    \V{C}^{-M}_{d} &                    &        &               \\
                   & \V{C}^{-(M-1)}_{d} &        &               \\
                   &                    & \ddots &               \\
                   &                    &        & \V{C}^{M}_{d} 
  \end{array}\right]
  \:
  \left[\begin{array}{c}
    \V{a}^{-M}\\
    \V{a}^{-(M-1)}\\
    \vdots\\
    \V{a}^{M}
  \end{array}\right]
$$
where each matrix $\V{C}^{n}_{d} \in \R^{1 \times (M-n+1)}$ for $n = -M,\ldots,M$ realizes the Clenshaw algorithm acting on the subvector $\V{a}^n$ of $\V{a}$.
The matrices $\V{C}^{n}_{d}$ can be further decomposed corresponding to the number of steps in the Clenshaw algorithm by
\begin{eqnarray*}
  \V{C}^{n}_{d}       & := & \V{C}_{d,1}^{n} \: \V{C}_{d,2}^{n} \: \ldots \: \V{C}_{d,M-\abs{n}}^{n},\\
  \V{C}_{d,l}^{n}     & := & \encl{[}{\V{I}_{l}, \V{\tilde{e}}_{d,l}}{]} \in \R^{l \times (l+1)},\\
  \V{\tilde{e}}_{d,l} & := & \paren{0,0,\ldots,\gamma_{k}^n,\alpha_{k}^n \vtheta_{d} + \beta_{k}^n}^{\transp} \in \R^{l} \quad \paren{l=1,\ldots,M-\abs{n}}.
\end{eqnarray*}
Finally, we compute the sum
$$
  f_{d} = \sum_{n=-M}^M b^{n}_{d} \e^{\im n \vphi_{d}}
$$
by the scalar product
$$
  f_{d} 
  = 
%  \V{E}_{d}
%  \:
%  \V{a}
%  =
  \paren{\e^{\im (-M) \vphi_{d}},\e^{\im (-(M-1)) \vphi_{d}},\ldots,\e^{\im M \vphi_{d}}}
  \:   
  \left(\begin{array}{c}
    b^{-M}_{d}\\
    b^{-(M-1)}_{d}\\
    \vdots\\
    b^{M}_{d}
  \end{array}\right).
$$
This gives the representation
$$
  \V{E}_{d} = \paren{\e^{\im (-M) \vphi_{d}},\e^{\im (-(M-1)) \vphi_{d}},\ldots,\e^{\im M \vphi_{d}}} \:  
  \left[\begin{array}{cccc}
    \V{C}^{-M}_{d} &                    &        &               \\
                   & \V{C}^{-(M-1)}_{d} &        &               \\
                   &                    & \ddots &               \\
                   &                    &        & \V{C}^{M}_{d} 
  \end{array}\right].
$$

\section{Adjoint and Inverse DSFT}

We obtain the sought algorithm for the adjoint transform directly from the factorization given in the last section. We have to evaluate
$$
  \V{\tilde{a}}
  :=
  \V{Y}^{\h} \V{f}
  = 
  \encl{[}{
    \V{E}_{0}^{\h},
    \V{E}_{1}^{\h},
    \ldots,
    \V{E}_{D-1}^{\h}
  }{]}
  \:
  \left(\begin{array}{c}
    f_{0}\\
    f_{1}\\
    \vdots\\
    f_{D-1}
  \end{array}\right). 
$$
This decomposition implies the evaluation of the matrix-vector product in a nonstandard way by traversing the 
matrix column-wise instead of row-wise. After each column $\V{E}_{d}^{\h}$, the result vector is updated with the portion due 
to the product $\V{\tilde{a}}_{d} := \V{E}^{\h}_{d} \: f_{d}$. We obtain
\begin{equation}
  \nonumber
  \V{\tilde{a}} = \V{\tilde{a}}_{0} + \V{\tilde{a}}_{1} + \ldots + \V{\tilde{a}}_{D-1}
\end{equation}
and
\begin{equation}
  \label{NFSFT:adjointNDSFTFactorization}
  \V{\tilde{a}}_{d}
  =
  \left[\begin{array}{cccc}
    {\V{C}^{-M}_{d}}^{\transp} &                               &        &                           \\
                              & {\V{C}^{-(M-1)}_{d}}^{\transp} &        &                           \\
                              &                                & \ddots &                           \\
                              &                                &        & {\V{C}^{M}_{d}}^{\transp} 
  \end{array}\right]
  \:
  \left(\begin{array}{c}
    \e^{\im M \vphi_{d}}\\
    \e^{\im (M-1) \vphi_{d}}\\
    \vdots\\
    \e^{\im (-M) \vphi_{d}}
  \end{array}\right)
  \:
  f_{d}.
\end{equation}
A multiplication with a transposed matrix ${\V{C}^{n}_{d}}^{\transp}$ realizes a \emph{"transposed" Clenshaw algorithm} summarized in Algorithm \ref{NFSFT:transposedClenshaw}. The complete algorithm can be obtained from \eqref{NFSFT:adjointNDSFTFactorization} directly and is given in Algorithm \ref{NFSFT:adjointDSFT}.
\begin{algorithm}[htb]
  \caption{Transposed Clenshaw Algorithm}
  \label{NFSFT:transposedClenshaw}    
  \begin{algorithmic}
    \STATE  Input: $M \in \NZ$, $n \in \NZ$ with $\abs{n} \le M$, $d$, $\tilde{b}^n_{d}$
    \STATE
    \STATE $\tilde{a}_{d,\abs{n}-1}^n := 0$
    \STATE $\tilde{a}_{d,\abs{n}}^n := \tilde{b}^n_{d}$
    \FOR {$k=\abs{n}+1,\ldots,M$} 
      \STATE $\tilde{a}_{d,k}^n := \paren{\alpha_{k}^n\vtheta_{d} + \beta_{k}^n}\tilde{a}_{d,k-1}^n + \gamma_{k}^n \tilde{a}_{d,k-2}^n$
    \ENDFOR
    \STATE
    \STATE Output: $\V{\tilde{a}}^n_{d}$
\end{algorithmic}
\end{algorithm}
\begin{algorithm}[htb]
  \caption{Adjoint DSFT}
  \label{NFSFT:adjointDSFT}    
  \begin{algorithmic}
    \STATE  Input: $M \in \NZ$, $D \in \N$, $\mathcal{X}$, $\fun{\V{f}}{\mathcal{X}}$
    \STATE
    \FOR {$n=-M,\ldots,M$} 
      \FOR {$k=\abs{n},\ldots,M$} 
        \STATE $\tilde{a}_{k}^n := 0;$
      \ENDFOR
    \ENDFOR
    \FOR {$d=0,\ldots,D-1$} 
      \FOR {$n=-M,\ldots,M$} 
        \STATE $\tilde{b}^n_{d} := f_{d} \: \e^{\im n \vphi_{d}}$
        \STATE Compute $\V{\tilde{a}}^n_{d}$ by the transposed Clenshaw algorithm
        \STATE $\V{\tilde{a}}^n := \V{\tilde{a}}^n + \V{\tilde{a}}^n_{d}$
      \ENDFOR
    \ENDFOR
    \STATE
    \STATE Output: $\V{\tilde{a}}$
\end{algorithmic}
\end{algorithm}

We have now obtained an $\bigo{M^2\:D}$ algorithm for the multiplication with $\V{Y}^{\h}$. Based on \ref{NFSFT:InversionFormula} and algorithm \ref{NFSFT:adjointDSFT} we obtain $\bigo{M^4}$ algorithms \ref{} and \ref{} for computing the Fourier coeffcients $a_{k}^n$ from the function samples $\fun{\V{f}}{\mathcal{X}^{\gl}}$ and $\fun{\V{f}}{\mathcal{X}^{\cc}}$. We refer to these algorithms as the \emph{direct inverse discrete spherical Fourier transform} of \emph{Gauss-Legendre type (iDSFT-GL)} and \emph{Clenshaw-Curtis type (iDSFT-CC)}, respectively.

\begin{algorithm}[htb]
  \caption{Direct iDSFT-GL}
  \label{NFSFT:directIDSFTGL}    
  \begin{algorithmic}
    \STATE Input: $M \in \NZ$, $\fun{\V{f}}{\mathcal{X}^{\gl}}$
    \STATE
    \FOR {$j = 0,\ldots,2M+1$}
      \FOR {$l = 0,\ldots,M$}
        \STATE $f_{j,l} = w^{\gl}_{l} f_{j,l}$ 
      \ENDFOR
    \ENDFOR
    \STATE Compute $\V{a}$ by an adjoint DSFT
    \STATE
    \STATE Output: $\V{a}$
  \end{algorithmic}
\end{algorithm}
\begin{algorithm}[htb]
  \caption{Direct iDSFT-CC}
  \label{NFSFT:directIDSFTCC}    
  \begin{algorithmic}
    \STATE Input: $M \in \NZ$, $\fun{\V{f}}{\mathcal{X}^{\cc}}$
    \STATE
    \FOR {$j = 0,\ldots,2M+1$}
      \FOR {$l = 0,\ldots,2M$}
        \STATE $f_{j,l} = \varepsilon_{l}^{2M} w^{\cc}_{l} f_{j,l}$ 
      \ENDFOR
    \ENDFOR
    \STATE Compute $\V{a}$ by an adjoint DSFT
    \STATE
    \STATE Output: $\V{a}$
  \end{algorithmic}
\end{algorithm}

The algorithms for the computation of the products $\fun{\V{f}}{\mathcal{X}} = \fun{\V{Y}}{\mathcal{X}} \: \V{a}$ and $\V{\tilde{a}} = \fun{\V{Y}}{\mathcal{X}}^{\h} \: \fun{\V{f}}{\mathcal{X}}$ presented so far have both computational complexity $\bigo{M^2\:D}$ making computations for most applications impracticably slow. We already mentioned that FFT techniques may be applied to sums of the form
$$
  f_{d} := \sum_{n = -M}^M b^n_d \e^{\im n \vphi_{d}} \quad \paren{d = 0,\ldots,D}
$$
if the nodes $\vphi_d$ are distributed uniformly. For a nonuniform distribution one may apply the NFFT algorithm from Section \ref{Basics:NFFT}. In the following section we present a fast algorithm for the evaluation of the sums $b^{n}_{d}$ from \eqref{NFSFT:directDSFT:firstSum} first mentioned in \cite{postta97}. To be more precise, the algorithm performs for a fixed $n$ a change of basis independent of the nodes to transform a sum of the form
\begin{equation}
  \label{NFSFT:LegendreSum}
  \sum_{k=\abs{n}}^M a_{k}^n \fun{P_{k}^n}{x}
\end{equation}
into a representation by Chebyshev polynomials. This is possible since the associated Legendre functions $P_{k}^n$ are (up to a factor $\frac{1}{\sqrt{1-x^2}}$ for odd $n$) polynomials of degree at most $M$. Once obtained this representation and taking into account that the argument $x$ takes the form $\cos \vtheta$, one may rewrite the sum again as an ordinary Fourier sum that can be evaluated by the NFFT algorithm at nonequispaced nodes in a fast way, too. In total this results in an approximative $\bigo{M^2 \log^2 M + mD}$ algorithm, where $m$ is a constant depending on the desired accuracy. 

%This holds only in the case of equispaced nodes. If one leaves 
%In this chapter, we first derive an algorithm for the fast evaluation of a bandlimited function $f$ at arbitrary nodes. Due to the separability 
%of the basis functions $Y_k^n$, evaluating the sum in \eqref{NFSFT:FourierExpansion} can be split up in ordinary discrete Fourier transforms for nonequispaced 
%data in colatitudinal direction and \emph{discrete Legendre function transforms} for the latitudinal part. The basic idea of the approach 
%presented here is to perform a change of basis from Legendre functions to complex exponentials in latitudinal direction to get the 
%representation
%$$ \fun{f}{\vtheta,\vphi} = \sum_{n=-M}^{M} \sum_{k=-M}^M c_k^n e^{ik\vtheta} e^{in\vphi}.$$
%The computation of the function values $f_j := \fun{f}{\vtheta_j,\vphi_j}$ can now be performed 
%using a two-dimensional NFFT. 

%Once derived a fast algorithm for this product, which implies a factorization of $\V{Y}$ into a product of sparse matrices,
%one also immediately gets from the theoretical point of view a fast algorithm for the adjoint product 
%$$\V{a} = \V{Y}^{\h} \; \V{f}.$$

\section{Fast Legendre Function Transform}
\label{DSFT:FLFT}

In this and the following section, we assume $n \ge 0$ without loss of generality. We consider polynomials of the form
\begin{equation}
  \label{NFSFT:GnEven}
 \fun{g^n}{x} := \sum_{k=n}^M a_k^n \fun{P_k^n}{x} \in \Pol_M
 \end{equation}
for even $n$ and
\begin{equation}
  \label{NFSFT:GnOdd}
  \fun{g^n}{x} := \frac{1}{\sqrt{1-x^2}} \sum_{k=n}^M a_k^n \fun{P_k^n}{x} \in \Pol_{M-1}
\end{equation}
for odd $n$ with given complex coefficients $a_k^n$.
The \emph{fast Legendre tunction transform (FLFT)} performes a change of basis and computes the 
coefficients $b_k^n$ of the Chebyshev representation
$$ \fun{g^n}{x} = \sum_{k=0}^M b_k^n \fun{T_k}{x}$$
for even $n$ and
$$\fun{g^n}{x} = \sum_{k=0}^{M-1} b_k^n \fun{T_k}{x}$$
for odd $n$.

Lemma \ref{Basics:AssociatedLegendreRecurrence} implies
\begin{equation}
  \label{NFSFT:U}
  \left(\begin{array}{c}
    P_{c+k}^{n} \\ P_{c+k+1}^{n}
  \end{array}\right)
  =
  \fun{U_{k}^{n}}{\cdot,c}^{\transp}\;
  \left(\begin{array}{c}
    P_{c-1}^{n} \\ P_{c}^{n}
  \end{array}\right)
\end{equation}
where
$$
  \fun{U_{k}^{n}}{\cdot,c}^{\transp} :=
  \left(\begin{array}{cc}
    \gamma_c^{n} \fun{P_{k-1}^{n}}{\cdot,c+1} & \gamma_c^{n} \fun{P_{k}^{n}}{\cdot,c+1} \\
                 \fun{P_{k}^{n}}{\cdot,c}     &              \fun{P_{k+1}^{n}}{\cdot,c}
  \end{array}\right).   
$$
We set $a_k^n := 0$ for $k < n$ or $k > M$. In a first step, we use Lemma 
\ref{Basics:AssociatedLegendreRecurrence} to write
$$ g^n = \sum_{k = n}^{N-1} a_{k}^{(0)} P_k^{n}$$
with
\begin{eqnarray*}
  \fun{a_k^{(0)}}{x}     & := & a_k^n \quad (k = 0,\ldots,N-3),\\
  \fun{a_{N-2}^{(0)}}{x} & := & a_{N-2}^n + \gamma_{N-1}^{n} a_N^n,\\
  \fun{a_{N-1}^{(0)}}{x} & := & a_{N-1}^n + \paren{\alpha_{N-1}^{n}x + \beta_{N-1}^{n}} a_N^n.
\end{eqnarray*}
We now set 
$$\tilde{n} := \fun{\min}{n,N-2}, \quad \tilde{M} := \left\{\begin{array}{l@{\quad \text{if} \quad}l} M & M < N, \\ M-1 & M = N.\end{array}\right.$$
and obtain
$$
  g^n = \sum_{l = \floor{\frac{\tilde{n}}{4}}}^{\ceil{\frac{\tilde{M}+1}{4}}-1} \paren{\sum_{k=0}^{3} a_{4l+k}^{(0)} P_{4l+k}^{n}}.
$$

Notice that $a_k^{(0)}$ are polynomials of 
degree at most $1$ and that one immediately has their corresponding Chebyshev coefficients.

From \eqref{NFSFT:U} with $k = 1$ and $c = l+1$ it follows that
$$
\left(\begin{array}{c}
  P_{4l+2}^{n}, 
  P_{4l+3}^{n}
\end{array}\right)
\left(\begin{array}{c}
  a_{4l+2}^{(0)}\\
  a_{4l+3}^{(0)} 
\end{array}\right)
=
\left(\begin{array}{c}
  P_{4l}^{n},
  P_{4l+1}^{n}
\end{array}\right)
{\mathbf{U}_{1}^{n}\left(\cdot,4l+1\right)}
\left(\begin{array}{c}
  a_{4l+2}^{(0)}\\
  a_{4l+3}^{(0)} 
\end{array}\right)
$$
and therefore
$$ g^n = \sum_{l = \floor{\frac{\tilde{n}}{4}}}^{\ceil{\frac{\tilde{M}+1}{4}}-1} a_{4l}^{(1)} \paren{P_{4l}^{n} + a_{4l+1}^{(1)} P_{4l+1}^{n}} $$
with
\begin{equation}
\label{NFSFT:FirstStep}
  \left(\begin{array}{c}
    a_{4l}^{(1)}\\
    a_{4l+1}^{(1)} 
  \end{array}\right)
  =
  \left(\begin{array}{c}
    a_{4l}^{(0)}\\
    a_{4l+1}^{(0)} 
  \end{array}\right)
  + {\mathbf{U}_{1}^{n}\left(\cdot,4l+1\right)}
  \left(\begin{array}{c}
    a_{4l+2}^{(0)}\\
    a_{4l+3}^{(0)} 
  \end{array}\right).
\end{equation}
We can use Algorithm \ref{Basics:Algorithm:FastPolynomialMultiplication} with $K=2$ to compute 
the polynomial products in \eqref{NFSFT:FirstStep}. Applying this idea repeatedly leads to a cascade 
summation with steps $\tau=1,\ldots,t-1$ as illustrated in Figure \ref{NFSFT:Figure:CascadeSummation}. 
In step $\tau$ we compute
\begin{equation}
  \label{NFSFT:GeneralStep}
  \left(\begin{array}{c}
    a_{2^{\tau+1}l}^{(\tau)}\\
    a_{2^{\tau+1}l+1}^{(\tau)} 
  \end{array}\right)
  =
  \left(\begin{array}{c}
    a_{2^{\tau+1}l}^{(\tau-1)}\\
    a_{2^{\tau+1}l+1}^{(\tau-1)} 
  \end{array}\right)
  + {\fun{\V{U}_{2^{\tau}-1}^{n}}{\cdot,2^{\tau+1}l+1}}
  \left(\begin{array}{c}
    a_{2^{\tau+1}l+2^{\tau}}^{(\tau-1)}\\
    a_{2^{\tau+1}l+2^{\tau}+1}^{(\tau-1)} 
  \end{array}\right)
\end{equation}
for $\paren{l=\floor{\frac{\tilde{n}}{2^{\tau+1}}},\ldots,\ceil{\frac{\tilde{M}+1}{2^{\tau+1}}}-1}$
by applying Algorithm \ref{Basics:Algorithm:FastPolynomialMultiplication} with $N=2^{\tau}$. It can be shown that
$a_{2^{\tau+1}l}^{(\tau)} = a_{2^{\tau+1}l+1}^{(\tau)} \equiv 0$ for $l < \floor{\frac{\tilde{n}}{2^{\tau+1}}}$ and 
$l > \ceil{\frac{\tilde{M+1}}{2^{\tau+1}}}-1$.
After step $\tau = t-1$ we arrive at
\begin{equation}
  \nonumber
  g^n = a_{0}^{(t-1)} P_{0}^{n} + a_{1}^{(t-1)} P_{1}^{n}.
\end{equation}
Since 
\begin{equation}
  \nonumber
  \fun{P_{0}^n}{x} = \frac{\left( \left( 2n \right) ! \right)^{1/2}}{2^n n!},\ \fun{P_{1}^n}{x} = \left(\alpha_{0}^nx + \beta_{0}^n\right)\fun{P_{0}^n}{x}
\end{equation} 
we get
\begin{equation}
  \label{NFSFT:LastStep}
  g^n = \frac{\left( \left( 2n \right) ! \right)^{1/2}}{2^n n!} a_{0}^{(t-1)} + a_{1}^{(t-1)} \left(\alpha_{0}^nx + \beta_{0}^n\right)\fun{P_{0}^n}{x}
\end{equation}
Using 
\begin{equation}
  \nonumber
  xT_{0}(x) = T_{1}(x),\ xT_{k}(x) = \frac{1}{2}\left( T_{k+1}(x) + T_{k-1}(x) \right)
\end{equation}
we finally obtain the sought Chebyshev coeffcients $b_{k}^n$ of $g^n$ from \eqref{NFSFT:LastStep} by
applying elementary vector operations on the vectors of Chebyshev coefficients of $a_{0}^{(t-1)}$ and $a_{1}^{(t-1)}$ 
(see Section \ref{NFSFT:LinearAlgebra:LastStep}).

In total, the FLFT consists of $t+1 = \bigo{\log M}$ steps. The first and the last step clearly have computational complexity $\bigo{M}$. The rest of the steps is the cascade summation. Each step has computational complexity $\bigo{M \log M}$ due to the DCT applications in Algorithm \ref{Basics:Algorithm:FastPolynomialMultiplication} used for the multiplication with the matrices $U$. In total this accumulates to $\bigo{M \log^2 M}$ for the whole algorithm.

\begin{figure}
  \label{NFSFT:Figure:CascadeSummation}
  % Cascade summation
  \unitlength0.87cm
    \begin{picture}(14,14)
      % setze 6 Boxen
      \multiput(0,0)(0,2.5){6}{\framebox(14,1)[lb]}
      % schreibe in die 1. Box
      \multiput(1.64,12.5)(1.64,0){8}{\line(0,1){1}}
      %\put(14.4,12.8){\large $\in \mathbb R$}
      \put(0.3,12.8){\large $0$}
      \put(1.0,12.8){\large $0$}
      \put(1.94,12.8){\large $0$}
      \put(2.64,12.8){\large $0$}
      \put(3.59,12.8){\large $a_4^4$}
      \put(4.29,12.8){\large $a_5^4$}
      \put(5.24,12.8){\large $a_6^4$}
      \put(5.94,12.8){\large $a_7^4$}
      \put(6.88,12.8){\large $a_8^4$}
      \put(7.58,12.8){\large $a_9^4$}
      \put(8.45,12.8){\large $a_{10}^4$}
      \put(9.15,12.8){\large $a_{11}^4$}
      \put(10.1,12.8){\large $a_{12}^4$}
      \put(10.8,12.8){\large $a_{13}^4$}
      \put(11.75,12.8){\large $a_{14}^4$}
      \put(12.4,12.8){\large $a_{15}^4$}
      \put(13.3,12.8){\large $a_{16}^4$}
      % schreibe zwischen 2. und 3. Box
      \multiput(12.2,11.5)(1.3,0){2}{\line(0,1){1}}
      \put(12.2,11.5){\line(1,0){1.3}}
      \put(12.7,11){\line(0,1){0.5}}
      %\multiput(0.82,11)(1.64,0){7}{\line(0,1){1.5}}
      \multiput(4.12,11)(1.64,0){5}{\line(0,1){1.5}}
      % schreibe in die 2. Box
      \multiput(1.64,10)(1.64,0){7}{\line(0,1){1}}
      %\put(14.4,10.3){\large $\in \Pi_1$}
      \put(0.3,10.3){\large $0$}
      \put(1.0,10.3){\large $0$}
      \put(1.94,10.3){\large $0$}
      \put(2.64,10.3){\large $0$}
      \put(3.5,10.3){\large $a_4^{(0)}$}
      \put(4.2,10.3){\large $a_5^{(0)}$}
      \put(5.1,10.3){\large $a_6^{(0)}$}
      \put(5.8,10.3){\large $a_7^{(0)}$}
      \put(6.75,10.3){\large $a_8^{(0)}$}
      \put(7.45,10.3){\large $a_9^{(0)}$}
      \put(8.35,10.3){\large $a_{10}^{(0)}$}
      \put(9.05,10.3){\large $a_{11}^{(0)}$}
      \put(10.05,10.3){\large $a_{12}^{(0)}$}
      \put(10.7,10.3){\large $a_{13}^{(0)}$}
      \put(11.85,10.3){\large $a_{14}^{(0)}$}
      \put(13.00,10.3){\large $a_{15}^{(0)}$}
      % schreibe zwischen 2. und 3. Box
      %\multiput(0.82,9)(1.64,0){7}{\line(0,1){1}}
      \multiput(4.10,9)(1.64,0){5}{\line(0,1){1}}
      \put(12.8,9){\line(0,1){1}}
      %\multiput(0.82,9)(3.28,0){3}{\line(1,0){1.64}}
      \multiput(4.10,9)(3.28,0){2}{\line(1,0){1.64}}
      \put(10.66,9){\line(1,0){2.14}}
      %\multiput(1.64,8.5)(3.28,0){4}{\line(0,1){0.5}}
      \multiput(4.92,8.5)(3.28,0){3}{\line(0,1){0.5}}
      %\put(0.95,9.3){$ U_1^4(\, \cdot\, ,1)$}
      \put(4.23,9.3){$ U_1^4(\, \cdot\, ,5)$}
      \put(7.5,9.3){$ U_1^4(\, \cdot\, ,9)$}
      \put(10.9,9.3){$ U_1^4(\; \cdot\; ,13)$}
      % schreibe in die 3. Box
      \multiput(3.3,7.5)(3.3,0){3}{\line(0,1){1}}
      %\put(14.4,7.8){\large $\in \Pi_3$}
      \put(0.82,7.8){\large $0$}
      \put(2.2,7.8){\large $0$}
      \put(3.8,7.8){\large $a_4^{(1)}$}
      \put(5.5,7.8){\large $a_5^{(1)}$}
      \put(7,7.8){\large $a_8^{(1)}$}
      \put(8.8,7.8){\large $a_9^{(1)}$}
      \put(10.5,7.8){\large $a_{12}^{(1)}$}
      \put(12.5,7.8){\large $a_{13}^{(1)}$}
      % schreibe zwischen 3. und 4. Box
      \multiput(1.64,6.5)(3.28,0){4}{\line(0,1){1}}
      \put(1.64,6.5){\line(1,0){3.28}}
      \put(8.2,6.5){\line(1,0){3.28}}
      \multiput(3.3,6.0)(6.6,0){2}{\line(0,1){0.5}}
      \put(2.43,6.8){$ U_3^4(\; \cdot\; ,1)$}
      \put(9.07,6.8){$ U_3^4(\; \cdot\; ,9)$}
      % schreibe in die 4. Box
      %\put(14.4,5.5){\large $\in \Pi_7$}
      \put(6.5,5){\line(0,1){1}}
      \put(2,5.3){\large $a_0^{(2)}$}
      \put(4,5.3){\large $a_1^{(2)}$}
      \put(8.5,5.3){\large $a_8^{(2)}$}
      \put(10.5,5.3){\large $a_9^{(2)}$}
      % schreibe zwischen 4. und 5. Box
      \multiput(3.3,4.0)(6.6,0){2}{\line(0,1){1}}
      \put(3.3,4){\line(1,0){6.6}}
      \put(5.70,4.3){$ U_7^4(\; \cdot\; ,1)$}
      \put(6.5,3.5){\line(0,1){0.5}}
      \put(6.5,1){\line(0,1){1.5}}
      % schreibe in die 4. Box
      %\put(14.4,2.8){\large $\in \Pi_{15}$}
      %\put(14.4,0.3){\large $\in \Pi_{16}$}
      \put(3,2.8){\large $a_0^{(3)}$}
      \put(9.5,2.8){\large $a_1^{(3)}$}
      \put(5.85,0.3){\large $\paren{b_{k}^{n}}_{k=0}^{16}$}
    \end{picture}
  \caption{Schematic representation of the FLFT cascade summation for $M = 16$ and $n = 4$.}
\end{figure}

\begin{algorithm}[htb]
  \caption{Fast Legendre Function transform (unstabilized)}
  \label{NFSFT:Algorithm:FLFT_unstab}    
  \begin{algorithmic}
    \STATE Input:  $M \in \NZ$, $n \in \NZ$ $(n \le M)$, $\paren{a_k^n}_{k=n}^{M}$
    \STATE Precompute: $t$, $N$, $\tilde{n}$, $\tilde{M}$ as above and $$\fun{U_{2^{\tau}-1}^{n}}{\cos \frac{\paren{2s+1}\pi}{2^{\tau+2}}, 2^{\tau+1}l+1}$$ 
    \STATE \invisible{Precompute:} for $\tau = 1,\ldots,t-1$, $l = \floor{\frac{\tilde{n}}{2^{\tau+1}}},\ldots,\ceil{\frac{\tilde{M}+1}{2^{\tau+1}}}-1$ and $s = 0,\ldots,2^{\tau+1}$
    \STATE \invisible{Precompute:} 

    \STATE $a_{0}^{(t-1),\text{stab}} := a_{1}^{(t-1),\text{stab}} := 0$
    \STATE Compute $\paren{a_{k}^{(0)}}_{k=0}^{N-1}$ 
    \FOR {$\tau=1,\ldots,t$}
      \FOR {$l = \floor{\frac{\tilde{n}}{2^{\tau+1}}},\ldots,\ceil{\frac{\tilde{M}+1}{2^{\tau+1}}}-1$} 
        \STATE Compute $a_{2^{\tau+1}l}^{(\tau)}$ and $a_{2^{\tau+1}l+1}^{(\tau)}$ from $$a_{2^{\tau+1}l}^{(\tau-1)},\  
          a_{2^{\tau+1}l+1}^{(\tau-1)},\ a_{2^{\tau+1}l+2^{\tau}}^{(\tau-1)},\ a_{2^{\tau+1}l+2^{\tau}+1}^{(\tau-1)}$$ using 
          \eqref{NFSFT:GeneralStep} and Algorithm \ref{Basics:Algorithm:FastPolynomialMultiplication}
      \ENDFOR
    \ENDFOR
    \STATE Compute $a_{0}^{(t-1)} := a_{0}^{(t-1)} + a_{0}^{(t-1),\text{stab}}$, $a_{1}^{(t-1)} := a_{1}^{(t-1)} + a_{1}^{(t-1),\text{stab}}$
    \STATE Output: $\paren{c_{k}}_{k=0}^{2N-1}$
\end{algorithmic}
\end{algorithm}

\section{Stabilization}
\label{DSFT:Stabilization}
Algorithm \ref{NFSFT:Algorithm:FLFT_unstab} is exact in exact arithmetic. Unfortunately, it becomes unstable in finite precision for $n > 16$. The computation is subject to cancelations since the associated Legendre polynomials $\fun{P_{k}^n}{x,c}$ involved in the algorithm become very large for certain triples $\paren{k,n,c}$ and $\abs{x} \approx 1$ while at the same time the polynomials $a_{2^{\tau+1}l+2^{\tau}}^{(\tau-1)}$ and $a_{2^{\tau+1}l+2^{\tau}}^{(\tau-1)}$ exhibit arbitrary small values (see \cite{kupo02}). A simple but effective idea first developed in \cite{97} is to replace the ordinary multiplication steps by "stabilization" steps, whenever the values $\fun{P_{k}^n}{x,c}$ exceed a certain threshold. The multiplications with $\fun{U_{2^{\tau+1}}^n}{\cdot,2^{\tau+1}l+1}$ are replaced by multiplications with $\fun{U_{2^{\tau}(2l+1)-1}^n}{\cdot,1}$ fulfilling
$$
	\left(\begin{array}{c}
	  P_{2^{\tau+1}l+2^{\tau}}^{n}\\ 
	  P_{2^{\tau+1}l+2^{\tau}+1}^{n}
	\end{array}\right)
	=
	\V{U}_{2^{\tau}(2l+1)-1}^{n}\left(\cdot,1\right)^{\transp}
		\left(\begin{array}{c}
	  P_{0}^{n}\\
	  P_{1}^{n}
	\end{array}\right).
$$
This corresponds to taking the polynomials $a_{2^{\tau+1}l+2^{\tau}}^{(\tau-1)}$ and $a_{2^{\tau+1}l+2^{\tau}+1}^{(\tau-1)}$ out of the cascade and updating
$$
	\left(\begin{array}{c}
	  a_{0}^{(t-1)}\\
	  a_{1}^{(t-1)}
	\end{array}\right)^{\text{stab}}
	:=
	\left(\begin{array}{c}
	  a_{0}^{(t-1)}\\
	  a_{1}^{(t-1)}
	\end{array}\right)^{\text{old}}
  +
	\V{U}_{2^{\tau}(2l+1)-1}^{n}\left(\cdot,1\right)
	\left(\begin{array}{c}
	  a_{2^{\tau+1}l+2^{\tau}}^{(\tau-1)}\\
	  a_{2^{\tau+1}l+2^{\tau}+1}^{(\tau-1)}
	\end{array}\right)  
$$
after the cascade summation is completed. Clearly, each "stabilization" step can be implemented using Algorithm \ref{Basics:Algorithm:FastPolynomialMultiplication} 
again with $K = N$. Therefore, these steps are costly compared to the cascade summation without stabilization. We sacrifice runtime efficiency for accuracy. 
Unfortunately, an upper bound for the number $s$ of stabilization steps with respect to $M$ for a given threshold is not known yet. However, if $s = \bigo{\log M} $, 
we still have an $\bigo{M \log^2 M}$ algorithm. Our computations for $M$ up to $1024$ and various thresholds show (see Figure \ref{NFSFT:figure:stabilization}) 
that this might be indeed the case. 
\begin{figure}[htb]
  \centering
  \includegraphics[width=12cm]{images/stabilization}
  \caption{The relative number of stabilization steps for thresholds 1000 (blue), 10000 (red), 100000 (magenta) and 1000000 (black).}
  \label{NFSFT:figure:stabilization}
\end{figure}
Algorithm \ref{Fast Legendre Function transform (stabilized)} summarizes the complete algorithm with stabilization.
\begin{algorithm}[htb]
  \caption{Fast Legendre Function transform (stabilized)}
  \label{NFSFT:Algorithm:FLFT_stab}    
  \begin{algorithmic}
    \STATE Input:  $M \in \NZ$, $n \in \NZ$ $(n \le M)$, $\paren{a_k^n}_{k=n}^{M}$
    \STATE Precompute: $t$, $N$, $\tilde{n}$, $\tilde{M}$ as above and $$\fun{U_{2^{\tau}-1}^{n}}{\cos \frac{\paren{2s+1}\pi}{2^{\tau+2}}, 2^{\tau+1}l+1}$$ 
    \STATE \invisible{Precompute:} for $\tau = 1,\ldots,t-1$, $l = \floor{\frac{\tilde{n}}{2^{\tau+1}}},\ldots,\ceil{\frac{\tilde{M}+1}{2^{\tau+1}}}-1$, 
    \STATE \invisible{Precompute:} $s = 0,\ldots,2^{\tau+1}$ and the stabilization steps.
    \STATE $a_{0}^{(t-1),\text{stab}} := a_{1}^{(t-1),\text{stab}} := 0$
    \STATE Compute $\paren{a_{k}^{(0)}}_{k=0}^{N-1}$ 
    \FOR {$\tau=1,\ldots,t$}
      \FOR {$l = \floor{\frac{\tilde{n}}{2^{\tau+1}}},\ldots,\ceil{\frac{\tilde{M}+1}{2^{\tau+1}}}-1$} 
        \IF {multiplication with \fun{U_{2^{\tau}-1}^{\abs{n}}}{\cos \frac{\paren{2s+1}\pi}{2^{\tau+2}}, 2^{\tau+1}l+1} is stable}
          \STATE Compute $a_{2^{\tau+1}l}^{(\tau)}$ and $a_{2^{\tau+1}l+1}^{(\tau)}$ from $$a_{2^{\tau+1}l}^{(\tau-1)},\  
            a_{2^{\tau+1}l+1}^{(\tau-1)},\ a_{2^{\tau+1}l+2^{\tau}}^{(\tau-1)},\ a_{2^{\tau+1}l+2^{\tau}+1}^{(\tau-1)}$$ using 
            \eqref{NFSFT:GeneralStep} and Algorithm \ref{Basics:Algorithm:FastPolynomialMultiplication}
        \ELSE
          \STATE Update $a_{0}^{(t-1),\text{stab}}$ and $a_{1}^{(t-1),\text{stab}}$ from
            $$ 
              a_{2^{\tau+1}l+2^{\tau}}^{(\tau-1)},\ a_{2^{\tau+1}l+2^{\tau}+1}^{(\tau-1)}
            $$ 
            using \eqref{NFSFT:StabilizationStep} and Algorithm \ref{Basics:Algorithm:FastPolynomialMultiplication}
          \STATE Update $a_{2^{\tau+1}l^{(\tau)}} := a_{2^{\tau+1}l^{(\tau-1)}}$, $a_{2^{\tau+1}l+1}^{(\tau)} := a_{2^{\tau+1}l+1}^{(\tau-1)}$
        \ENDIF
      \ENDFOR
    \ENDFOR
    \STATE Compute $a_{0}^{(t-1)} := a_{0}^{(t-1)} + a_{0}^{(t-1),\text{stab}}$, $a_{1}^{(t-1)} := a_{1}^{(t-1)} + a_{1}^{(t-1),\text{stab}}$
    \STATE Output: $\paren{c_{k}}_{k=0}^{2N-1}$
\end{algorithmic}
\end{algorithm}

\section{A Linear Algebra Approach}
\label{DSFT:LinearAlgebra}

In this section we represent the FLFT algorithm as a linear mapping acting on a vector of Fourier coefficients $\V{a}^n$, hence a matrix. 
%Let again $n$ with $\abs{n} \le M$ be fixed. 
The FLFT can be represented as a matrix $\V{T} \in \R^{(N+1) \times (N+1)}$ that multiplied with a vector $\V{a}^n = \left(a_0^n,a_1^n,\dots,a_N^n\right)^T \in \C^{N+1}$ of Fourier coefficients gives a vector $\V{b}^n := \left(b_0^n,b_1^n,\dots,b_{2N-1}^n\right)^{\transp} \in \C^{2N}$ containing the Chebyshev coefficients $b_{k}^n$ of the polynomial $\fun{g^n}{x}$ from \eqref{NFSFT:GnEven} or \eqref{NFSFT:GnOdd}, respectively: 
$$\V{b}^{n} = \V{T} \; \V{a}^n.$$ 
For the sake of simplicity we omit the fact that $a_{k}^n = 0$ for $k < n$ or $k > M$ exploited in Algorithm \ref{NFSFT:Algorithm:FLFT_stab} in order to save some computational steps.

The algorithm implies a factorization of $\V{T}$ into sparse matrices that can be derived directly.
%from the algorithm already presented 
We will use this fact to obtain an algorithm for the transposed problem. In general, the FLFT consists of $t+1$ steps such that $\V{T}$ can be written as 
$$
  \V{T} = \V{T}^{(t)} \: \cdot \:  \V{T}^{(t-1)} \: \cdot \: \dots \: \cdot \: \V{T}^{(1)} \: \cdot \:  \V{T}^{(0)},
$$
with
$$
 \V{T}^{(\tau)} \in \left\{\begin{array}{l@{\quad \text{if} \quad}l} \R^{2N \times (N+1)} & \tau = 0, \\ \R^{2N \times 2N} & 1 \le \tau < t, \\ \R^{(N+1) \times 2N} & \tau = t. \end{array}\right.
$$

\subsection{The First Step}

The first step converts each Fourier coefficent $a_{k}^n$ into a polynomial of degree at most 1 in Chebyshev representation $\V{a}_{k}^{(0)} \in \C^2$. The result 
$$
  \V{a^{(0)}} := 
    \left[\begin{array}{c}
      \V{a}_{0}^{(0)}\\
      \vdots\\
      \V{a}_{N-1}^{(0)}
    \end{array}\right] 
    \in \C^{2N}
$$ 
is a vector of length $2N$, hence 
$$ 
  \V{a}_{k}^{(0)} = \V{e} \: a_{k}^n,\quad \text{with } \V{e} := \left(\begin{array}{l}1\\0\end{array}\right) \quad \paren{k = 0,\ldots,N-3}.
$$
The last polynomial $a_{N}^{(0)}$ is mapped to the preceeding two polynomials by means of the three-term recurrence for associated Legendre Functions, 
i.e. 
$$
  a_{N}^{(0)} = \left(\alpha x + \beta\right)a_{N-1}^{(0)} + \gamma a_{N-2}^{(0)}.
$$
Following this, $\V{T}^{(0)}$ can be written as 
$$
  \V{T}^{(0)} = \encl{[}{\V{I}_{N} \otimes \V{e},\;\V{\tilde{e}}}{]},
$$
with $\V{\tilde{e}} := \left(0,0,\dots,0,\gamma, 0, \beta,\alpha\right)^{\transp} \in \R^{2N}$.

\subsection{Steps $\tau = 1,\ldots,t-1$ (Cascade Summation)}
These steps represent the cascade summation applied to associated Legendre functions. In each round, half of the the functions is eliminated by mapping them to the remaining functions. Therefore, the vector $\V{a}^{(\tau-1)}$ is divided into consecutive blocks of four polynomials
$$
  \V{a}^{(\tau-1)} = 
    \left[\begin{array}{c}
      \V{a}^{(\tau-1)}_{0}\\
      \vdots\\
      \V{a}^{(\tau-1)}_{\frac{N}{2^{\tau+1}}-1}
    \end{array}\right]  \in \C^{2N},
$$
with
\begin{equation}
  \nonumber
  \V{a}^{(\tau-1)}_{l} := 
    \left[\begin{array}{l}
      \V{a}^{(\tau-1)}_{2^{\tau+1}l}\\[1ex]
      \V{a}^{(\tau-1)}_{2^{\tau+1}l+1}\\[1ex]
      \V{a}^{(\tau-1)}_{2^{\tau+1}l+2^{\tau}}\\[1ex]
      \V{a}^{(\tau-1)}_{2^{\tau+1}l+2^{\tau}+1}
    \end{array}\right] \in \C^{2^{\tau+2}} \quad \paren{l=0,\ldots,\frac{N}{2^{\tau+1}}-1},
\end{equation}
where $\V{a}^{(\tau-1)}_{2^{\tau+1}l+j} \in \C^{2^{\tau}}$ for $j = 0,1,2^{\tau},2^{\tau}+1$, representing the polynomial factors in front of each remaining function $P_{k}^n$. Each polynomial is represented by its vector of Chebyshev coefficients of length $2^{\tau}$. In every block, the first and the second polynomial, $a^{(\tau-1)}_{2^{\tau+1}l}$
and $a^{(\tau-1)}_{2^{\tau+1}l+1}$ remain unchanged. The third and the fourth polynomial, $a^{(\tau-1)}_{2^{\tau+1}l+2^{\tau}}$ and $a^{(\tau-1)}_{2^{\tau+1}l+2^{\tau}+1}$, are multiplied with the matrix $\fun{\V{U}_{2^{\tau}-1}^{n}}{\cdot,2^{\tau+1}l+1}$ which transforms them into a representation in terms of the first two functions. Following this, the output contains only half of the polynomials, but due to the multiplication with $\fun{\V{U}_{2^{\tau}-1}^{n}}{\cdot,2^{\tau+1}l+1}$ the degree might double each time so that twice the space is needed to store the Chebyshev coefficients. In total, the result vector still has length $2N$. 
%For each step $1 \le \tau < t$ and for each block $$\mb{\tilde{a}}_{l}^{(\tau-1)} := \left(\mb{a}_{4l}^{(\tau-1)},\mb{a}_{4l+1}^{(\tau-1)},\mb{a}_{4l+2}^{(\tau
%-1)},\mb{a}_{4l+3}^{(\tau-1)}\right)^T \text{, where } 0 \le l < 2^{t-\tau-1},$$ 
We need to keep the first two polynomials, but with their vectors zero-padded to twice the length. Furthermore, we have to add the coefficients due to the multiplication of the third and fourth polynomial with the matrix $\fun{\V{U}_{2^{\tau}-1}^{n}}{\cdot,2^{\tau+1}l+1}$.
Correspondingly, each block $\V{a}_{l}^{(\tau-1)}$ is multiplied by a matrix $\V{V}^{(\tau)}_l := \left[\V{Z^{(\tau)}},\V{U}^{(\tau)}_l\right]$, with
$$\V{Z}^{(\tau)} := \left(\begin{array}{cccc} \V{I}_{2^{\tau}} & 0\\ 0 & 0 \\ 0 & \V{I}_{2^{\tau}} \\ 0 & 0 \end{array}\right) \in \R^{2^{\tau+2} \times 2^{\tau+1}},\ \V{U}^{(\tau)}_l \in \R^{2^{\tau+2} \times 2^{\tau+1}}.$$
%Correspondingly, this can be written as the product 
%$$\mb{ZP}_{\tau} \; \mb{\tilde{a}}_{l}^{\tau-1} \text{, with } \mb{ZP}_{\tau} := \left(\begin{array}{cccc} \mb{I}_{2^{\tau}} & 0 & 0 & 0\\ 0 & 0 & 0 & 0 \\ 0 & \mb{I}_{2^{\tau}} & 0 & 0 \\ 0 & 0 & 0 & 0 \end{array}\right) \in \R^{2^{\tau+2} \times 2^{\tau+2}}.$$ 
%The multiplication with the matrix $\mb{U}$ that acts on the third and fourth polynomial is written as $\mb{U}_{\tau}^l \; \mb{\tilde{a}}_{l}^{\tau-1}$ where the matrix can be factorized as follows:
The matrix $\V{U}^{(\tau)}_l \in \R^{2^{\tau+2} \times 2^{\tau+1}}$ can be factorized as follows:
\begin{equation} 
  \label{NFSFT:LinearAlgebra:Factorization}
  \V{U}^{(\tau)}_l := \V{D}^{(\tau)}_{\text{II}} \; \cdot \; \V{S}^{(\tau)} \; \cdot \; \fun{\V{P}^{(\tau)}}{2^{\tau + 1}l+1} \; \cdot \; \V{D}^{(\tau)}_{\text{III}}
\end{equation}
where we define
\begin{align*}
  \V{D}^{(\tau)}_{\text{II}} & := \V{I}_{2} \otimes \left(\V{D}_{2^{\tau+1}} \V{C}_{2^{\tau+1}}\right) & \in \R^{2^{\tau+2} \times 2^{\tau+2}},\\
  \V{S}^{(\tau)} & := \V{I}_2 \otimes \left[\begin{array}{cc}\V{I}_{2^{\tau+1}},\V{I}_{2^{\tau+1}}\end{array}\right] & \in \R^{2^{\tau+2} \times 2^{\tau+3}},\\
  \fun{\V{P}^{(\tau)}}{c} & := \text{diag}\left(\gamma_{c}^n \fun{\V{P}_{2^{\tau}-2}^n}{c+1},\gamma_{c}^n \fun{\V{P}_{2^{\tau}-1}^n}{c+1},\right.\\
    & \left. \hspace{9ex} \fun{\V{P}_{2^{\tau}-1}^n}{c}, \fun{\V{P}_{2^{\tau}}^n}{c}\right) & \in \R^{2^{\tau+3} \times 2^{\tau+3}},\\
  \V{D}^{(\tau)}_{\text{III}} & := \V{I}_{2} \otimes \left(\left(\V{I}_{2} \otimes \V{C}^{\transp}_{2^{\tau+1}}\right)\V{Z}^{(\tau)}\right) & \in \R^{2^{\tau+3} \times 2^{\tau+1}}.
\end{align*}   
This follows directly from Algorithm \ref{NFSFT:Algorithm:FLFT_unstab}. The matrix $\V{D}^{(\tau)}_{\text{III}}$ realizes first the zero-padding ($\V{Z}^{(\tau)}$)
of the two polynomials $a^{(\tau-1)}_{2^{\tau+1}l+2^{\tau}}$ and $a^{(\tau-1)}_{2^{\tau+1}l+2^{\tau}+1}$, second the evaluation at the Chebyshev nodes ($\V{\tilde{C}}^{\transp}$) and finally a duplication of the result vector in order to permit multiplication with two different associated Legendre polynomials for each polynomial. The matrix $\fun{\V{P}^{(\tau)}}{2^{\tau+1}l+1}$ contains on its main diagonal the associated Legendre polynomials of the matrix $\fun{\V{U}_{2^{\tau}-1}^n}{\cdot,2^{\tau+1}l+1}$ evaluated at the Chebyshev nodes. It realizes therefore a pointwise multiplication with the zero-padded and evaluated polynomials $a^{(\tau-1)}_{2^{\tau+1}l+2^{\tau}}$ and $a^{(\tau-1)}_{2^{\tau+1}l+2^{\tau}+1}$. The matrix $\V{S}^{(\tau)}$ forms the sums for the two rows of the result and finally the $\V{D}^{(\tau)}_{\text{II}}$ transforms the new obtained polynomials $a^{(\tau)}_{2^{\tau+1}l}$ and $a^{(\tau-1)}_{2^{\tau+1}l+1}$ back into Chebyshev representation.

From the factorization in \eqref{NFSFT:LinearAlgebra:Factorization} the more compact representation
\begin{equation}
\label{UCompact}
\V{U}^{(\tau)}_l = 
\left(\begin{array}{lclrcr}
  \V{D}_{2^{\tau+1}} \V{C}_{2^{\tau+1}} & \gamma_{c}^n & \left(\right. & \fun{\V{P}_{2^{\tau}-2}^n}{c+1} \V{C}^{\transp}_{2^{\tau+1}} \V{Z}_{1} & + & 
  \fun{\V{P}_{2^{\tau}-1}^n}{c+1} \V{C}^{\transp}_{2^{\tau+1}} \V{Z}_{2} \left.\right) \\
  \V{D}_{2^{\tau+1}} \V{C}_{2^{\tau+1}} & & \left(\right. & \fun{\V{P}_{2^{\tau}-1}^n}{c} \V{C}^{\transp}_{2^{\tau+1}} \V{Z}_{1} & + & 
  \fun{\V{P}_{2^{\tau}}^n}{c} \V{C}^{\transp}_{2^{\tau+1}} \V{Z}_{2} \left.\right)
\end{array}\right)
\end{equation}
is obtained.
%So for each block $l$, a multiplication with a matrix $\mb{V}_{\tau}^l  \in \R^{2^{\tau+2} \times 2^{\tau+2}}$ is applied where
%$$ \mb{V}_{\tau}^l = \left[\mb{ZP},\mb{U}_{\tau}^l\right].$$ 
The complete step can finally be represented as $$\V{T}^{(\tau)} = \text{diag}\left(\V{V}^{(\tau)}_0,\V{V}^{(\tau)}_1,\dots,\V{V}^{(\tau)}_{2^{t-\tau-1}-1}\right).$$

\subsubsection{The Last Step}
\label{NFSFT:LinearAlgebra:LastStep}
The last step consists in calculating the polynomial $g^{n} = a_{0}^{(t-1)} P_{0}^{n} + a_{1}^{(t-1)} P_{1}^{n}$ in Chebyshev representation. Since 
$$
  \fun{P_{0}^n}{x} = \frac{\left( \left( 2n \right) ! \right)^{1/2}}{2^n n!},\ \fun{P_{1}^n}{x} = \left(\alpha_{0}^n x + \beta_{0}^n\right)P_{0}^n(x),
$$ 
we can use 
$$
  xT_{0}(x) = T_{1}(x),\ xT_{k}(x) = \frac{1}{2}\left( T_{k+1}(x) + T_{k-1}(x) \right)
$$ 
to write
$$ 
  \V{g}^{n} = \gamma_{0}^n \left( \V{I}_{N+1} \V{a}_{0}^{(\tau-1)} + \left( \alpha_{0}^n\V{W}_{N+1} + \beta_{0}^n\V{I}_{N+1} \right) \V{a}_{1}^{(\tau-1)} \right).
$$
Here, we have defined
$$
\mb{W}_{k} :=
\left(
\begin{array}{ccccccc}
  0 & \frac{1}{2} &             &                           \\
  1 &           0 & \frac{1}{2} &                           \\
    & \frac{1}{2} &           0 & \ddots                    \\
    &             &      \ddots & \ddots      & \frac{1}{2} \\
    &             &             & \frac{1}{2} &           0
\end{array}
\right)
\in \R^{k \times k} \quad \paren{k \in \N}.
$$
Depending on $n \in \NZ$, we can distinguish three cases:
\begin{description}
  \item[n odd] In this case, $\alpha_{0}^n = 0$ and $\beta_{0}^n = 1$ so that $\V{T}^{(\tau)}$ is 
    $$
      \V{T}^{(\tau)} = \gamma_{0}^n \left[ \V{I}_{N+1}, \V{I}_{N+1} \right].
    $$
  \item[n = 0] Here it holds, $\alpha_{0}^n = 1$ and $\beta_{0}^n = 0$ and we get $$\V{T}^{(\tau)} = \gamma_{0}^n \left[ \V{I}_{N+1}, \V{W}_{N+1} \right].$$
  \item[n even, n > 0] Now $\alpha_{0}^n = -1$ and $\beta_{0}^n = 1$ which results in $$\V{T}^{(\tau)} = \gamma_{0}^n \left[ \V{I}_{N+1}, \V{I}_{N+1} - \mb{W}_{N+1} \right].$$
\end{description}

%\subsection{The Adjoint Operator}
%Following the factorization of $\mb{T}$ given in the previous section, one obtains easily the adjoint operator which is paramount for an implementation of xxx: 
%$$\mb{T}^H = \mb{T}_{0}^H \; \cdot \; \mb{T}_{1}^H \dots \mb{T}_{t-1}^H \; \cdot \; \mb{T}_{t}^H.$$ For $\tau = 0$ and $\tau = t$ we obtain immediatly
%$$ \mb{T}_{0}^H = \left( \begin{array}{c} \mb{I}_{N} \otimes \mb{e}_{1}^T\\ \mb{\tilde{e}}^T \end{array}\right), \mb{T}_{t}^H = \gamma_{0}^n \left\{\begin{array}{l@{\quad \text{if} \quad}l} 
% \left[ \begin{array}{c} \mb{I}_{N+1} \\ \mb{I}_{N+1} \end{array} \right] & \text{n odd},\\[2ex]
% \left[ \begin{array}{c} \mb{I}_{N+1} \\ \mb{T}_{N+1}^T \end{array} \right] & \text{n = 0},\\[2ex]
% \left[ \begin{array}{c} \mb{I}_{N+1} \\ \mb{I}_{N+1} - \mb{T}_{N+1}^T \end{array} \right] & \text{n even, n > 0}.
%\end{array}\right.$$
%Using xxx we get for the rest of the steps $\mb{T}_{\tau}$, $1 \le \tau \le t-1$
%\begin{eqnarray*}
% \mb{T}_{\tau}^H & = & \text{diag}\left({\mb{V}_{\tau}^0}^H,{\mb{V}_{\tau}^1}^H,\dots,{\mb{V}_{\tau}^{2^{t-\tau-1}-1}}^H\right),\\
% {\mb{V}_{\tau}^l}^H & = & \left[ \begin{array}{c} \mb{Z}^H \\ { \mb{U}_{\tau}^l}^H \end{array} \right],\\
% {\mb{U}_{\tau}^l}^H & = &
%   \left(
%     \begin{array}{rlllr}
%        \gamma_{c}^n & \left(\right. Z_{1}^T \mb{\tilde{C}}_{2^{\tau+1}} \mb{P}_{2^{\tau}-2}^n(c+1)   & + & Z_{2}^T \mb{\tilde{C}}_{2^{\tau+1}} \mb{P}_{2^{\tau}-1}^n(c+1)
%         & \left.\right) \mb{\tilde{C}}_{2^{\tau+1}}^T \\
%        & \left(\right. Z_{1}^T \mb{\tilde{C}}_{2^{\tau+1}} \mb{P}_{2^{\tau}-1}^n(c) & + & Z_{2}^T \mb{\tilde{C}}_{2^{\tau+1}} \mb{P}_{2^{\tau}}^n(c) & \left.\right) \mb{\tilde{C}}_{2^{\tau+1}}^T
%     \end{array}
%   \right).
%\end{eqnarray*}

%
%\begin{figure}[tb]
%  \centering
%  \includegraphics[height=9cm,width=12cm]{images/accuracy}
%  \caption{To be written...}
%  \label{NFSFT:Figure:Accuracy}
%\end{figure}


\section{Adjoint Fast Spherical Fourier Transform}
\label{DSFT:AdjointTransform}