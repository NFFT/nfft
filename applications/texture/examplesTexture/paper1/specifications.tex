\documentclass[a4paper,twoside,smallheadings,headsepline,11pt,final]{scrartcl}

\usepackage[automark]{scrpage2}

% Ams Math-packages
\usepackage{amsmath}
\usepackage{amsthm}
\usepackage{amssymb}
\usepackage{latexsym}

\usepackage[utf8]{inputenc}
\usepackage[T1]{fontenc}

% Graphics for figures
\usepackage{graphicx}
\usepackage{xcolor}

\pagestyle{scrheadings}
\rehead[\today]{\today}
\lohead[\today]{\today}

\newcommand{\N}{\mathbb{N}}
\newcommand{\F}{\mathcal{F}}
\newcommand{\T}{\mathcal{T}}
\newcommand{\C}{\mathbb{C}}
\renewcommand{\S}{\mathbb{S}}

\title{Specifications for the Numerical Tests}
\author{Matthias Schmalz}

\begin{document}

\maketitle

\section{General Requirements}

\begin{itemize}
\item Starting point of the CGNE and CGNR algorithm is $\mathbf{0}$.
\item If possible, I will execute the algorithm on machines having the same hardware components.
In this case, I will document processor type and speed, memory size and operation system.
\item I will use the current revision of the nfft (directory: lib/trunk) and document the revision number.
\item The "final product" will be a CD containing the documented source code for the tests (and Nfft library), all randomly generated data, and reports generated by the test programs.
\end{itemize}

\section{Numerical Tests}
\subsection{A Single Test}

During a single test, we evaluate some given Fourier coefficients at some given nodes.
This yields some \emph{samples}.
We use these samples to reconstruct the Fourier coefficients with the solver of the Nfft3 library.
Output of the test are the \emph{relative residual norm}, the \emph{relative $L^2(\mathrm{SO}(3))$ error norm} and the \emph{relative weighted error norm}.

\paragraph*{Input:}
\begin{itemize}
\item bandwidth $L$, Fourier coefficients $\mathbf{\hat P}_{\mathrm{ref}} \in \C^{J_L}$,
\item a set of nodes $\Gamma \subset \S^2 \times \S^2$,
\item algorithm: CGNE or CGNR.
\end{itemize}

\paragraph*{Execution:}
\begin{enumerate}
\item Compute samples $\mathbf{P} := \F_{L, \Gamma}(\mathbf{\hat P}_{\mathrm{ref}})$.
\item Apply the algorithm (CGNE or CGNR) to compute either
\[\mathbf{\hat P} := \T_{L, \Gamma}^E(\mathbf{P})\] or
\[\mathbf{\hat P} := \T_{L, \Gamma}^R(\mathbf{P}).\]
The damping factors are planned to be chosen as $1$;
however, if the convergence behaviour is not good, I will try some damping factors.
Weights are $1$ (default values), and I do not apply regularisation;
i.\,e. the solvers minimizes either
\[\|\mathbf{P} - \F_{L, \Gamma}(\mathbf{\hat P})\|_2\quad \text{(CGNR)}\] or
\[\|\mathbf{\hat P}\|_{\hat W} \text{ subject to } \F_{L, \Gamma}(\mathbf{\hat P}) = \mathbf{P}\quad \text{(CGNE)}.\]
The solver is stopped if the residual becomes very small, worse, or does not improve anymore.
\end{enumerate}

\paragraph*{Output:}
The output consists of the following \emph{error measures}:
\begin{itemize}
\item \emph{relative residual norm:}
$\mathrm{rrn} := \dfrac{\|\mathbf{P} - \F_{L, \Gamma}(\mathbf{\hat P})\|_2}{\|\mathbf{P}\|_2}$,
\item \emph{relative $L^2(\mathrm{SO}(3))$ error norm:}
$\mathrm{ren} := \dfrac{\|\mathbf{\hat P}_{\mathrm{ref}} - \mathbf{\hat P}\|_{M^{-2}}}{\|\mathbf{\hat P}_{\mathrm{ref}}\|_{M^{-2}}}$,
\item \emph{relative weighted error norm:}
$\mathrm{rwen} := \dfrac{\|\mathbf{\hat P}_{\mathrm{ref}} - \mathbf{\hat P}\|_{\hat W}}{\|\mathbf{\hat P}_{\mathrm{ref}}\|_{\hat W}}$.
\end{itemize}

\subsection{The Testsuite}

The testsuite invokes the single test with several bandwidths, several sets of nodes and several choices of the algorithm (CGNE or CGNR).
If the testsuite would use only one vector of Fourier coefficients (for a given bandwidth), then the results could depend too much on this designated vector of Fourier coefficients.
Therefore, the testsuite invokes the single test with several (randomly generated) vectors of Fourier coefficients and computes the mean values (or some other statistic) of the error measures ($\mathrm{rrn}$, $\mathrm{ren}$ or $\mathrm{rwen}$) concerning different vectors of Fourier coefficients.
The output of the testsuite consists of various plots visualising these \emph{accumulated} error measures.

\paragraph*{Input:}
\begin{itemize}
\item the number of testcases $\mathit{cases}$ (planned: 100),
\item a scaling constant $\mathit{maxnumber}$ (planned: $2^{31}-1$),
\item several bandwidths $L$ (planned: $L = 5, 10, 20, 40, 80$),
\item several sets of nodes $\Gamma$ (planned: $\Gamma = \{(h_i, r_j)\ |\ 1 \leq i \leq N,\ 1 \leq j \leq N'\}$, $N = 11, 23, 41, 92, 164, 308$, $N' \approx N^2$, the $h_i$ are approximatively equidistributed on the hemispere, $r_j$ are approximatively equidistributed on the sphere.
\end{itemize}

\paragraph*{Execution:}
\begin{enumerate}
\item Randomly generate and store $\{[\mathbf{\hat P}_{\mathrm{ref}}]_n\}_{n = 1}^{\mathit{cases}} \subset \C^{J_{L_{\mathrm{max}}}}$.
The real and imaginary parts of the complex numbers $[\mathbf{\hat P}_{\mathrm{ref}}]_n(\ell, k, k')$ are independently rectangular distributed w.\,r.\,t.\ 
\[[-\mathit{maxnumber} / \ell^3, \mathit{maxnumber} / \ell^3].\]
("The $[\mathbf{\hat P}_{\mathrm{ref}}]_n$ are randomly chosen with decay $\ell^3$.")
\item For each bandwidth $L$, set of nodes $\Gamma$, algorithm (CGNE or CGNR):
\begin{enumerate}
\item For each $n$ with $1 \leq n \leq \mathit{cases}$:
\begin{enumerate}
\item Invoke a single test with $L$, $\mathbf{\hat P}_{\mathrm{ref}} := [\mathbf{\hat P}_{\mathrm{ref}}]_n \Bigr|_{J_L}$, $\Gamma$ and the algorithm (CGNE or CGNR).
\item Save the output of the single test in $\mathrm{rrn}_n, \mathrm{ren}_n, \mathrm{rwen}_n$.
\end{enumerate}
\item Compute the mean value $\overline{\mathrm{rrn}} := \dfrac{1}{\mathrm{cases}}\sum_{n=1}^{\mathrm{cases}} \mathrm{rnn}_n$ (and maybe other statistics).
\item Procede analogously for $\{\mathrm{ren}_n\}_n$ and $\{\mathrm{rwen}_n\}_n$, respectiveley.
\end{enumerate}
\end{enumerate}

\paragraph*{Output:}
The output consists of several figures and plots related to the different algorithms (CGNE and CGNR) and error measures ($\mathrm{rrn}$, $\mathrm{ren}$ or $\mathrm{rwen}$).
There is, e.\,g., a figure related to CGNE and $\mathrm{rrn}$.
This figure contains for each node set $\Gamma$ a plot consisting of the points $(L, \overline{\mathrm{rrn}})$, where $L$ ranges over the considered bandwidths and $\overline{\mathrm{rrn}}$ over the related mean values of the saved outputs $\{\mathrm{rrn}_n\}_n$.

\section{An Application to Texture Analysis}

\paragraph*{Input:}
\begin{itemize}
\item set of nodes $\Gamma$,
\item a bandwidth $L_{\mathrm{max}}$ (planned: $256$), a vector of Fourier coefficients $\mathbf{\hat f} \in \C^{J_{L_{\mathrm{max}}}}$ (planned: randomly chosen with decay $\ell^{2.5}$),
\item several bandwidths $L$ with $L \leq L_{\mathrm{max}}$ (planned: $L = 0, 2, 4, \mathbf{8}, \dots, 128$).
\end{itemize}

\paragraph*{Execution:}
\begin{enumerate}
\item Compute $\mathbf{\hat P}_{\mathrm{ref}}(\ell, k, k') := \mathrm{even}(\ell) \mathbf{\hat f}(\ell, k, k')$.
\item Compute $\mathbf{P} := \F_{L_{\mathrm{max}}, \Gamma}(\mathbf{\hat P})$.
\item For each bandwidth $L$:
\begin{enumerate}
\item Use the CGNR algorithm to compute $\mathbf{\hat P} := \T_{L, \Gamma}^R(\mathbf{P})$.
The damping factors are planned to be chosen as $1$;
however, if the convergence behaviour is not good, I will try some damping factors.
Weights are $1$ (default values), and I do not apply regularisation;
The solver is stopped if the residual becomes very small, worse, or does not improve anymore.
\item For each $\ell$ with $0 \leq \ell \leq L$, compute
\[
\varepsilon(L, \ell) := 
\dfrac
{\sqrt{\sum_{k, k' = -\ell}^{\ell} |\mathbf{\hat f}(\ell, k, k') - \mathbf{\hat P}(\ell, k, k')|^2}}
{\sqrt{\sum_{k, k' = -\ell}^{\ell} |\mathbf{\hat f}(\ell, k, k')|^2}}.
\]
\end{enumerate}
\end{enumerate}

\paragraph*{Output:}
For each $L$ there is a designated plot.
This plot consists of the points $(\ell, \varepsilon(L, \ell))$, where $\ell = 0, 2, 4, 6, \dots, L$.

\section{Notations}

In this section I illustrate the differences between the notations in my "Studienarbeit" and in our paper.

\begin{tabular}{|l|l|}
\hline
Studienarbeit &  Paper\\ \hline \hline
$\phi$ & $\rho$\\ \hline
$Y_{\ell}^n(x) = P_{\ell}^{|n|}(\cos \theta) e^{i n \phi}$ & 
$Y_{\ell}^k(\xi) = \sqrt{\dfrac{2\ell + 1}{4 \pi}} P_{\ell}^{|k|}(\cos \theta) e^{i n \rho}$ \\ \hline 
$N$ & $L$ \\ \hline
$\mathbf{x} := \left[\T_N^{h, r}(\omega)\right]_{i,j} =$ & $\mathbf{P} := \left[\F_{L, \Gamma}(\mathbf{\hat P})\right]_{i,j} =$\\ $\sum_{(\ell,m,n) \in J_N} \omega_{\ell,m,n} \overline{Y_{\ell}^n(h_i)} Y_{\ell}^m(r_{i,j})$ & 
$\sum_{(\ell,k,k') \in J_L} \mathbf{\hat P}_{\ell,k,k'} Y_{\ell}^{k'}(h_i) \overline{Y_{\ell}^k(r_{i,j})}$\\ \hline
$\mathbf{x} = \overline{\mathbf{P}}$ &
$\mathbf{P} = \overline{\mathbf{x}}$ \\ \hline
$\omega = \dfrac{2\ell + 1}{4 \pi} \overline{\mathbf{\hat P}}$ & 
$\mathbf{\hat P} = \dfrac{4 \pi}{2\ell + 1} \overline{\omega}$ \\ \hline
\end{tabular} 

\end{document}
